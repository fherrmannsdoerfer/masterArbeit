\chapter{Related work}
There are many groups all over the world that developed their own software to work on STORM images. This chapter shows the related work to give a better understanding what other concepts are used.\newline
The DAOSTORM algorithmn \cite{DAO} adepts algorithms from software that is used to investigate crowded stellar fields. For spot localisation a fit of multiple PSFs is used. This is done to small clusters of molecules and not globaly. A model PSF is automatically generated from single PSFs in the data. DAOSTORM runs on the CPU.\newline
FPGA Estimator \cite{simulated} is an algorithm developed from a group at the Kirchoff-Institut f\"ur Physik in Heidelberg. It provides background subtraction by smoothing the pixels intensity over time, assuming Poisson distributed intensities. For high density data a methode is used that sets all further pixel to zero once a local minimum is detected. A gaussian estimator is used to determine the parameters of the Gaussian. The estimated scale can be compared with the given scale.\newline
A maximum likelihood estimate of the PSFs parameters is used by GPUgaussMLE \cite{alg3}. For background subtraction either a constant threshold or a dark imaged aquired from all frames is used. Two filters of different size are applied to the data to determin signal candidates. A patch around the candidates is then used for maximum likelihood estimation of the PSFs parameters. Based on the estimated parameters filters are applied that compare this values with given input values for the PSF width. Also the uncertainty of the estimation was used to filter the candidates. This algorithm runs on the GPU.  And can also be applied to 3d data.\newline
An other algorithm is called M2LE \cite{M2LE} which uses an user defined threshold for candidate selection and the median of a roi for background suppression, an user specified ellipticity threshold which can be made intensity dependend to discard candidates with an too high ellipticity. The PSFs position is determined using a maximum likelihood estimater, separated Gaussians are used for speed up. \newline
The ImageJ plugin PeakFit \cite{peakFit} uses a 2d Gaussian non-linear least squares Levenberg-Marquardt fitting, the candidates are aquired subtracting two filters of different size. The PSFs width can either be specified or estimated from the data. The fitted spots are filtered based on their signal-to-noise ratio. For high density data multiple Gaussians are fitted to the data. As an ImageJ plugin it can deal with any kind of data processable via ImageJ.\newline
PYME \cite{PYME} is an python package with functionality that can be used for localisation microscopy and other forms of widefield microscopy. Besides the STORM data analysis it also provides reconstruction and postprocessint algorithms. It is applicable to 2d and 3d data. For noise treatment a Gaussian-Poisson noise mixture model is used.\newline
