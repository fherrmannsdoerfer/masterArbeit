Mikroskopie ist ein wichtiges Werkzeug der Zell-Biologie. Die H\"ochstaufl\"osende STORM Mikroskopie gew\"ahrt genauere Einblicke in Zellen als andere Arten der Lichtmikroskopie, \"uber die Abbe'sche Aufl\"osungsgrenze hinaus. Dies wird durch die Aufnahme vieler Bilder mit jeweils wenigen und damit leicht voneinander trennbaren Bildpunkten erreicht.\newline
SimpleSTORM ist ein effizienter Rekonstruktions-Algorithmus von STORM Daten, der von \cite{MAJoachim} entwickelt und im Rahmen dieser Masterarbeit weiterentwickelt wurde.\newline 
Für die Weiterentwicklung lag der Fokus, neben der Verbesserung des Algorithmus im Allgemeinen, auf der Vereinfachung der Bedienbarkeit im Besonderen. Durch Selbstkalibrierung ist die Angabe von Parametern wie dem Verst\"arkungsfaktor der Kamera oder der Breite der Punktantwort des Signals nicht mehr erforderlich. Diese werden automatisch bestimmt, k\"onnen aber auch vom Benutzer eingegeben werden. Dies erleichtert auch unerfahrenen Benutzern die Arbeit mit SimpleSTORM.\newline
Die Modellierung des Hintergrunds erlaubt es, Inhomogenit\"aten des Hintergrunds zu korrigieren und Aussagen \"uber die G\"ute der Signale zu treffen.\newline

Eine weitere Komponente der SimpleSTORM Software ist der Colorcomposer. Dieser erm\"oglicht die automatische Registrierung mehrerer Farbkan\"ale. Die Bestimmung der Kolokalisation zwischen verschiedenen Kan\"alen erlaubt die Untersuchung von Wechselwirkungen zwischen einzelnen Molek\"ulen.