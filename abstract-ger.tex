Mikroskopie ist ein wichtiges Werkzeug der Zell-Biologie. Die H\"ochstaufl\"osende STORM Mikroskopie gew\"ahrt genauere Einblicke in Zellen, \"uber die Abbe'sche Aufl\"osungsgrenze hinaus. Dies wird durch die Aufnahme vieler Bilder mit jeweils weniger und damit leichter voneinander trennbaren Bildpunkten erreicht.\newline
SimpleSTORM ist ein effizienter Algorithmus der von \cite{MAJoachim} entwickelt wurde und in dieser Masterarbeit weiterentwickelt wurde. Dabei lag der Fokus neben der Verbesserung des Algorithmus im Allgemeinen auf der Vereinfachung der Bedienung. Die Kernpunkte sind hierbei eine Selbskalibrierung und die Modellierung des Hintergrunds.\newline
Durch die Selbstkalibrierung m\"ussen keine Parameter vorgegeben werden, diese werden von dem Algorithmus selbst bestimmt. Dies erleichter es noch unerfahrenen Benutzern mit SimpleSTORM zu arbeiten.
Die Modellierung des Hintergrunds erlaubt es Inhomogenit\"aten des Hintergrunds zu korrigieren Aussagen \"uber die G\"ute der Signale zu treffen.

Die weiterentwickelte Colorcomposer Software erm\"oglicht die automatische Registrierung mehrerer Farbkan\"ale und die Bestimmung der Kolokalisation zwischen verschiedenen Kan\"alen erlaubt die Untersuchung von Wechselwirkungen zwischen einzelnen Molek\"ulen.