Mikroskopie ist ein wichtiges Werkzeug der Zell-Biologie. Die H\"ochstaufl\"osende STORM Mikroskopie gew\"ahrt genauere Einblicke in Zellen als andere Arten der Lichtmikroskopie, \"uber die Abbe'sche Aufl\"osungsgrenze hinaus. Dies wird durch die Aufnahme vieler Bilder mit jeweils wenigen und damit leicht voneinander trennbaren Bildpunkten erreicht.\newline
SimpleSTORM ist ein effizienter Algorithmus zur Rekonstruktion von STORM Daten, der von \cite{MAJoachim} entwickelt wurde und im Rahmen dieser Masterarbeit weiterentwickelt wurde.\newline 
Dabei lag der Fokus neben der Verbesserung des Algorithmus im Allgemeinen, auf der Vereinfachung der Bedienbarkeit. Durch Selbstkalibrierung ist die Angabe von Parametern wie dem Verst\"arkungsfaktor der Kamera oder der Breite der Punktantwort des Signal nicht mehr erforderlich, diese werden automatisch bestimmt, k\"onnen aber auch vom Benutzer eingegeben werden. Dies erleichter erleichtert auch unerfahrenen Benutzern die Arbeit mit SimpleSTORM.\newline
Die Modellierung des Hintergrunds erlaubt es Inhomogenit\"aten des Hintergrunds zu korrigieren und Aussagen \"uber die G\"ute der Signale zu treffen.\newline

Eine weitere Komponente ist die Colorcomposer Software. Diese erm\"oglicht die automatische Registrierung mehrerer Farbkan\"ale. Die Bestimmung der Kolokalisation zwischen verschiedenen Kan\"alen erlaubt die Untersuchung von Wechselwirkungen zwischen einzelnen Molek\"ulen.