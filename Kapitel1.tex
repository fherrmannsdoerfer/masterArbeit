\chapter{Introduction}
How does a virus reproduce itself? Which proteins are essential for neuro transmitters? How do proteins interact with each other? These questions and many more arise in biology or related sciences. Microscopy is the most powerfull tool to answer these questions. But light microscopy is limited in spatial resolution due to the diffraction limited, as described by \cite{Abbe}. Recently there were developed different methodes to increase the resolution of light microscopy beyond the diffraction limit, like photoactivated localization microscopy (PALM) \cite{Palm} or stochastic optical reconstruction microscopy (STORM) \cite{Storm}. This techniques use many images with sparsely distributed signals, comming from flourophors and are blured by diffraction, to determine their center with a sub pixel accuracy, instead of one image with all signals together which would be blured and impossible to find the true position of the flourophors.
\newline
STORM can also be used to investigate the distribution of different proteins within a cell. Therefore each protein is labeled with different flourophores. Images can be aquired showing just signal from one kind of flourophore. But to be able to seperate the different signals from the flourophores the emission spectrum must be destinct. This leads to cromatic aberration which results in images that are distorted and thus can't be aligned easily. 
Many people have developed their own software to process STORM data sets. But most of these programs usual need many of parameters that must be set so that it is difficult for someone who is not familiar with image processing or does not know the parameters to use this software right from the beginning.\newline

How to build a software that is easy to use even with no prior information about the data or knowledge of image processing? \newline

SimpleStorm is a software that calibrates itself. It estimates the camera parameters and the width of the point spread function of the flourophores. In contrast to many other software applications no threshold is needed.\newline
If more than one kind of fluorophore was used to stain the sample, the resulting images are distorted relative to each other. The colorcomposer tool was developed to do this task automatically and to analyse the aligned images. The colocalisation of the molecules of the different channels is a useful measure in biology to determine whether or not near molecules might interact. The colorcomposer software calculates both global and local measures of localisation.


