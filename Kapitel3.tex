\chapter{Multicolor registration}
\section{Background}
In microscopy it is often desirable to label different structures in a cell with
different colors. To do so our collaborators use different fluoroscent molecules
that emit light at different and distinguishable wavelengths. Using different
filters it is possible to capture pictures just containing light emited from one
fluorophore. To get a mulit-channel picture the different channels must be
aligned. Because different flourophores emit different wavelengths, cromatic
aberration apears. This means that the light for the same spot but with
different wavelenghts is not mapped to the same spot in the image. To align the
different channels despite cromatic aberration, beads are used. Beads are
flourophores added to the probe, that emit light in all wavelengths the
different markers do and therefore are visible in all channels. The beads can be
used as landmarks, because their position in the original image is at the same
spot. The task is to find a transformation that maps corresponding beads on each
other.
\section{Features of the colercomposer application}
\subsection{Automatic bead detection}
\subsection{Alignment of two multicolor images}
\subsection{Heatmap 

\section{Bead detection}
The input for the colorcomposer application is a text file created by the storm
algorithm that contains information about the position, intensity, symmetry,
framenumber and signal-to-noise ratio of each detection. The beads should
ideally be visible in most of the images, this means one must search for
detections that appear in almost every frame at the same position. Therefore it
is plausible to take every detection of the first 50 frames as initial
candidates for beads. After that candidates that are closer than a threshold are
merged to get a list of all location where beads might be. Given that list every
other detection is tested to belong to one of the bead candidates. If a
candidate gets too few members it is no longer considered to be a bead and
removed from the list.
\section{Align Beads}
After the beads for each channel are found the next task is to find the same
bead in each channel. It can happen that some beads occure in just one channel,
if this is the case there will be no corresponding bead in the other
channels.\\
To do so, the minimal number of beads, three to four, that are neccesary to
calculate the transformation are chosen randomly from the first channel. After that, based on
a probabilistic approach and a distance matrix containing information about the
distances between all beads of the two channels, three to four beads from the
second channel are chosen.\\
Using this pairs of beads a linear transformation is found like described by
\cite{MAJoachim}. Using this transformation it is tested how many beads match in
total. It is assumed that the correct transformation will match other bead pairs
that were not chosen to calculate this transformation. After that the whole
procedure is done multiple times. In the end the best transformation is
chosen.\\
In principle shearing should also be allowed for a linear transformation, but tests
indicate that shearing does not occure, so it is disabled to improve stability. If there are just
three beads in each channel, then every time a perfect transformation is found,
but with the constraint of forbidden shearing, the right solution can be
identified.

\section{Accuracy of Registration}

\section{Colocalisation}
\subsection{Global colocalisation}
\subsection{Local colocalisation}
\subsection{Validation of colocalisation approaches}
"Image set CBS001RGM-CBS010RGM from the Colocalization Benchmark Source
(www.colocalization-benchmark.com) was used to validate colocalization."
