\chapter{Related work}
There are many groups all over the world that have developed their own software to work on localization microscopy. This chapter shows the related work to give a better understanding what other concepts are used.\newline
There are several algorithms that estimate the PSF from the data:\newline

The DAOSTORM algorithmn (\cite{DAO}) adapts algorithms from a software that is used to investigate crowded stellar fields. For spot localization a fit of multiple PSFs is used. This is done to small clusters of molecules and not globaly. A model PSF is automatically generated from single PSFs in the data. DAOSTORM runs on the CPU.\newline
FPGA Estimator (\cite{simulated}) is an algorithm developed from a group at the Kirchoff-Institut f\"ur Physik in Heidelberg. It provides background subtraction by smoothing the pixels intensity over time, assuming Poisson distributed intensities. For high density data a method is used that sets all further pixel to zero once a local minimum is detected. A gaussian estimator is used to determine the parameters of the Gaussian. The estimated scale can be compared with the given scale.\newline
QuickPALM (\cite{quickpalm}) uses the methods from rapidSTORM (\cite{rapidstorm}) which was published earlier. RapidSTORM uses a Spalttiefpass filter and the H\"ogborn "CLEAN" method for background suppression and a to select the local maxima, a Gaussian function with scales either given or estimated from the data is fitted. An amplitude threshold is used to distinguish between signal and background pixels. 
A maximum likelihood estimate of the PSFs parameters is used by GPUgaussMLE (\cite{alg3}). For background subtraction either a constant threshold or a dark imaged aquired from all frames is used. Two filters of different size are applied to the data to determin signal candidates. A patch around the candidates is then used for maximum likelihood estimation of the PSFs parameters. Based on the estimated parameters filters are applied that compare these values with given input values for the PSF width. Also the uncertainty of the estimation was used to filter the candidates. This algorithm runs on the GPU and can also be applied to 3d data.\newline

Most of the algorithms use a maximum likelihood estimator to localise the maxima. An other example that uses the same method as simpleSTORM is called M2LE (\cite{M2LE}) which uses an user defined threshold for candidate selection and the median of a roi for background suppression, an user specified ellipticity threshold which can be made intensity dependend to discard candidates with an too high ellipticity. The PSFs position is determined using a maximum likelihood estimator, separated Gaussians are used for speed up. \newline

There are algorithms that provide their functionality as plugins for ImageJ or python packages. This makes it easier to implement this methods in ones workflow.\newline
There is tThe ImageJ plugin PeakFit (\cite{peakFit}) that uses a 2d Gaussian non-linear least squares Levenberg-Marquardt algorithm for fitting. The candidates are aquired by subtracting two filters of different sizes. The PSFs width can either be specified or estimated from the data. The fitted spots are filtered based on their signal-to-noise ratio. For high density data multiple Gaussians are fitted to the data. As an ImageJ plugin it can deal with any kind of data processable via ImageJ.\newline
PYME (\cite{PYME}) is a python package with functionalities that can be used for localization microscopy and other forms of widefield microscopy. Besides the STORM data analysis it also provides reconstruction and postprocessing algorithms. It is applicable to 2d and 3d data. For noise treatment a Gaussian-Poisson noise mixture model is used.\newline

