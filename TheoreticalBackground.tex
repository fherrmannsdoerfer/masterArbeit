\chapter{Theoretical background}
This chapter gives a brief overview over the most important distributions and transformations that are used for the SimpleSTORM algorithm. Also an introduction to the most important parts of a regular Charge-coupled Device (CCD) cameras is given. There is also a section about the data the algorithm is designed for.
\section{Distributions}
\subsection{Poisson distribution}
A very important probability distribution in physics is the Poisson
distribution. Its probability mass function is:
\begin{equation}
	p(n,\mu) = \frac{\mu^n}{n!}\exp(-\mu)
\end{equation}
The larger the mean value $\mu$ gets the more likely the Poisson distribution gets to a Gaussian distribution with a mean and a variance of $\mu$. Figure \ref{poisgaussdistr} shows Poisson and Gaussian distributions with different parameters. The mean value was shifted by 0.5 which results from continuity correction.\newline
Poisson distributions describe the results of ``counting experiments'' and are
therefore important for image processing as pictures taken with a
camera are in principle counts of photons reaching the camera.  \newline
Poisson distributions are defined for integer values only and its variance is
the same as the mean value of the distribution.\newline
The median of a Poisson distribution is approximatively given by
\begin{align}
	\text{median Pois}(\lambda) \approx \lambda + \frac{1}{3} - \frac{1}{50\lambda}
\end{align}
Another important attribute is
the skewnes which describes the assymetry.
\begin{align}
 \text{skewnes Pois}(\lambda) = \frac{1}{\sqrt{\lambda}}
\end{align}

\begin{figure}
\centering
\includegraphics[width = 0.9\textwidth]{pictures/poissgaussdistr.png}
	\caption{Different Poisson and Gaussian distrubutions. Especially for small numbers the Paisson and Gaussian distribution differ. The larger $\mu$ gets for the Poisson distribution the more similar it gets to a Gaussian with mean $\mu$ and sigma $\sqrt{\mu}$. The Poisson distributions were interpolated between their defined values.}
	\label{poisgaussdistr}
\end{figure}

\subsection{Skellam distribution}
The probability
mass function of the Skellam distribution is a function of the difference between
two Poisson random variables
\begin{equation}
	p(k;\mu_1, \mu_2) =
	\exp(-(\mu_1+\mu_2))\left(\frac{\mu_1}{\mu_2}\right)^{k/2}~I_{|k|}\left(2\sqrt{\mu_1
	\mu_2}\right)
\end{equation}  
$n_1$ and $n_2$ are the Poisson random variables and $k = n_1 - n_2$.
$I_{|k|}$ is the modified Bessel function of the first kind.\\
Mean $\mu$ and variance $\sigma$ of the Skellam distribution are given by
\begin{align}
	&&\mu &= \mu_1 - \mu_2,& \sigma^2 &= \mu_1 + \mu_2\\
	\Rightarrow &&\mu_1& = \frac{\mu + \sigma^2}{2},& \mu_2 &=\frac{-\mu +
	\sigma^2}{2}
\end{align}
Figure \ref{skellamdist} shows two Poisson distributions ($P_\text{red}$ and $P_\text{blue}$) and the resulting Skellam distribution (puple) of the difference $P_\text{blue} - P_\text{red}$.
\begin{figure}
\centering
\includegraphics[width = 0.7\textwidth]{pictures/skellamdist.png}
	\caption{Two Poisson distributions and the resulting Skellam distribution. The distributions were interpolated between the integer values.}
	\label{skellamdist}
\end{figure}



\section{The data}
\subsection{Labeling}
The concept of direct stochastic optical reconstruction microscopy (dSTORM) \cite{heilemann} is, to label structures of interest with fluorophors that can be exited using a laser with the appropriate wavelength directly, in contrast to STORM techniques which need an additional activator fluorophore. A short time after the activation the fluorophors emit a photon and fall back to the unexited state or lose the ability to get exited, they bleach out.\newline
The technique to stain the samples that was used for all real-world images in this thesis is called immunofluorescence. For this technique antibodies are used to attach the fluorescenic molecules to the samples. Antibodies target specific biomolecules within a cell that show their antigen. There are two classes of immunofluorescence techniques.\newline
Primary immunofluorescence uses only one antibody to that the fluorophore is directly attached.\newline
For secundary immunofluorescence two kinds of antibodies are used. One primary antibody that binds to the molecule or structure of interest and a second antibody that binds to the primary and carries the fluorophore.
\subsection{Description of the data sets}
The datasets for dSTORM microscopy that we recieve from our collaborators from
Bioquant are big datasets of several gigabyte in the Andor .sif format. Each
file conains a stack of pictures, normaly between 1000 and 10000, taken
consecutively with exposure times between 20 and 200 milliseconds.\newline

Figure \ref{rawStorm} shows a typical frame of raw data. In each frame there might be multiple fluorophores visible at the same time. Due to the large magnification, beyond the diffraction limit, the almost pointlike fluorophores appear as gaussian shaped signals, their point spread functions. The fluorophores are either attached to the biological structures that are of interest or they form a cluster, called a bead.\newline

Beads are larger and brighter than spots from only one fluorophore and are used to align multiple channels in the postprocessing step. Beads are designed to show up in every frame of the sequence at the same position and can therefore be used as landmarks for the alignment. They are composed of fluorophores of different colors to be visible in every channel.\newline

The other spots, bound to some proteins for example, only light up for a very short time. This is the key aspect of STORM. Instead of one frame that shows all fluorophores at the same time, thousands of frames are captured containing only a couple of point spread functions per frame. This gives the possibility to determine the center of each point spread function with a sub-pixel precision, and in the end when all points are displayed together in one picture, to an image with a resolution beyond the diffraction limit.

\begin{figure}
\centering
\includegraphics[width = 0.88\textwidth]{pictures/Pos2_2_red2-2frame2475Color.png}
	\caption{Raw image for dSTORM processing}
	\label{rawStorm}
\end{figure}

\section{CCD camera}
\subsection{Photon sources with shot noise}
The emission of photons is a random process that occures at unpredictible times. Therefore the number of photons passing through a plane is never constant but varies around some average value. The phenomena, that one can never determine exactly how many photons should hit the sensor chip of a CCD camera for example, is called shot noise. It playes a major role if the total number of photons is low, as from dark sources or with short exposure times of the camera.
\subsection{Quantum efficiency}
Quantum efficiency describes the fraction of photons that creates a detectable electron in a sensor chip. The quantum efficiency dependends on the wavelenght of the incoming photon. Photons with energies below the band gap can not produce a free electron that can be detected. The maximum of the quantum efficiency is basically caused by two effects. The higher the photons energy the higher the kinetic energy of the freed electron, but the electron is absorbed farer away from the detector and will therefore recombinate with a electron hole more likely.
\subsection{Gain}
There are two different gain factors involved in the capturing process of a camera. First the electric signal for each pixel might be amplified. Second there is a gain factor that describes the proportionality between collected electrons and the digital number it is associated with.
\subsection{Readout noise}
The origin of readout noise is the amplifier. The amplification is never perfect, this means that the exact number of electrons at the end of the amplification has some variation around the expected linearly increased value. There might also be some random signals of the electronics that add to the "true" signal. The readout noise does not depend on the exposure time.
\subsection{Dark current noise}
The dark current noise is generated by the thermal movement of the atoms in the sensor chip. The movement of molecules and atoms dependends on the temperature of the material, therefore dark current noise strongly depends on the temperature of the chip and can be reduced by cooling. Dark current noise generates electrons in the bins of each pixel. Even with closed shutter it is constantly increasing with time and follows Poisson statistics.
\subsection{Quantisation}
The signal must fit into the output color depth. It has to be rounded or truncated to fit in. This process introduces errors that can be seen as additional noise that is dependent on the intensity of the signal. High intensities are disturbed less relative to low intensies.


\section{Transformations}
\subsection{Transformation to Poisson distributed signal} \label{trafoPoiss}
The images aquired from the camera do not show the real intensities
$I_\text{true}$, which result from the photon emission of the probe, but
transformed ones $I_\text{meas}$. Two reasons why the taken image
differs from the true image are considerd.\\
There is dark current. This means that even a picture taken with closed shutter
would get some intensity, even without any light hitting the cameras sensor chip. This is the result of the thermal movement of the atoms in the sensor chip
and can be reduced by cooling. The dark current noise adds an almost constant
value $o$ to the output signal.
Incoming photons create electrons via inner photoelectric effect. This electrons
are collected for each pixel and might be amplified to get the final result.
Assuming a linear relation between the number of incoming photons and the number
of electrons created and a linear amplifier results in a factor $g$. This factor
is multiplied with the number of photons captured during exposure time for each
pixel.\\
If the gain factor $g$ and the offset $o$ are known the true intensity, the
number of photons detected is:
\begin{equation}
	I_\text{true} = \dfrac{I_\text{meas}-o}{g}. \label{transtopoiss}
\end{equation}
\subsection{Anscombe transformation}
\label{trafoAnscombe}
The Anscombe transform \cite{anscombe} is used to transform a random variable with a Poisson
distribution into one with an approximatly constant standard deviation. The
transformation is defined as:
\begin{equation}
	A(x) = 2\sqrt{x+\frac{3}{8}}.
\end{equation}
As one can see in figure \ref{anscombe} the Anscombe transformations result has
for mean intensities greater than 4 a intensity independent standard deviation of
one.
\begin{figure}
	\centering
	\includegraphics[width = 0.5\textwidth]{pictures/anscombe.png}
	\caption{Standard deviation over mean intensities of different Poisson
	distributions}
	\label{anscombe}
	
\end{figure}

\section{Estimation of camera gain}
\subsection{Using Skellam parameter} \label{skellam1}
The model assumes that the background and the beads are Poisson distributed. This data $I_\text{true}$ is then transformed in the following way:
\begin{equation}
	I_\text{meas} = g \cdot I_\text{true} + o \label{trafoGain}
\end{equation}
This is the inverse transformation of equatione \ref{transtopoiss}.\newline
Considering two pixels with different Poisson distributions $P_1$ and $P_2$ with mean value and variances of this distributions $\lambda_1$ and $\lambda_2$. If this distributions are transfomed as given in equation \ref{trafoGain} their mean and variance change like shown in equation \ref{meanvarPoiss1} and \ref{meanvarPoiss2}. This gives the oportunity to determine the gain and offset using two or more pixels.
\begin{align}
	\text{var}(I_{\text{meas}1})& = g^2\cdot\text{var}(I_{\text{true}1})\\ 
	\text{var}(I_{\text{meas}2})& = g^2\cdot\text{var}(I_{\text{true}2})\\
	\text{mean}(I_{\text{meas}1})& = g\cdot \text{mean}(I_{\text{true}1}) + o\\
	\text{mean}(I_{\text{meas}2})& = g\cdot \text{mean}(I_{\text{true}2}) + o
\end{align}
The values for $\text{var}(I_{\text{meas}1}), \text{var}(I_{\text{meas}2})$, $\text{mean}(I_{\text{true}1})$ and $\text{mean}(I_{\text{true}2})$ can be calculated from the data and can be used to calculate the gain as follows:
\begin{align}
	\frac{\text{var}(I_{\text{meas}1})-\text{var}(I_{\text{meas}2})}{\text{mean}({\text{meas}1})-\text{mean}(I_{\text{meas}2})}&= \frac{g^2\cdot \lambda_1  - g^2\cdot \lambda_2 }{g\cdot \lambda_1 + o - (g\cdot \lambda_2+o)}\\
	& = \frac{g^2\cdot(\lambda_1-\lambda_2)}{g\cdot (\lambda_1-\lambda_2)}\\
	& = g
\end{align}
The offset $o$ can be calculated in the same way.
\begin{align}
	\text{mean}(I_{\text{meas}1}) - \frac{\text{var}(I_{\text{meas}1})}{g} &= g\cdot \lambda_1 + o - \frac{g^2\cdot\lambda_1}{g}\\
	&= o
\end{align}
If more than two points are used a straight line must be fitted through them. For more robustness the Skellam distribution instead of the variances is used. It is favorable to perform this calculation on a wide range of intensities for better results.
\subsection{Using skewness of poisson distribution}
For every pixel there is a set of multiple values in the set. This allows to
calculate the different parameters individually for each pixel. One can
calculate mean and variance of the measured intensities $I_\text{meas}(i,j)$ and
gets
\begin{align}
	\text{mean}(I_\text{meas}(i,j))& = g\cdot \text{mean}(I_\text{true}(i,j)) + o \label{meanvarPoiss1}\\
	\text{var}(I_\text{meas}(i,j))& = g^2\cdot\text{var}(I_\text{true}(i,j)) \label{meanvarPoiss2}
\end{align}
Assuming a Poisson distribution as the true intensity, mean and variance would
be the same. Unfortunately the mean true intensities are unknown and it is
not possible to determin $g$ and $o$ so far. For large mean Intensities $\mu$
the Poisson distribution becomes more and more similar to a Gauss distribution
with the same mean. However, for small means, the Poisson distribution is not
symmetric. The skewness $s_p$ of a Poission distribution is the inverse of the
square root of the mean $(\mu)^{-.5}$. It can also be directly
calculated from data
\begin{equation}
	s_p = \frac{1}{n}\sum_{i = 1}^n \left(\frac{x_i - \bar x}{\sigma}\right)^3
\end{equation}
The skewness is invariant to shift and multiplication with a constant. This
means that the transformation caused by the camera gain and the dark current
does not affect the skewness. This gives a third equation to solve for $g$ and
$o$.\\
This approach is strictly limited to, at least for background pixels, low intensities. Tests have shown that if the mean of the true Poisson distributin is higher than roughly 30, the skewness gives, due to noise, no stabel results. For data sets with bright background intensities it is not possible to determine the camera parameters in this way.