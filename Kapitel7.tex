\chapter{Theoretical background}

\section{Transformations}
\subsection{Transformation to Poisson distributed signal} \label{trafoPoiss}
The images aquired from the camera show not the real intensities
$I_\text{true}$, which result from the photon emission of the probe, but
transformed ones $I_\text{meas}$. I consider two main reasons why the taken image
differs from the true image, besides noise.\\
There is dark current which means that even a picture taken with closed shutter
would get some intensity, even without any light hitting the sensor chip of the
camera. This is a result of thermal movement of the atoms off the sensor chip
and can be reduced by cooling. The dark current noise adds an almost constant
value $o$ to the output signal.
Incoming photons create electrons via inner photoelectric effect. This electrons
are collected for each pixel and might be amplified to get the final result.
Assuming a linear relation between the number of incoming photons and the number
of electrons created and a linear amplifier results in a factor $g$. This factor
is multiplied with the number of photons captured during exposure time for each
pixel.\\
If the gain factor $g$ and the offset $o$ are known the true intensity, the
number of photons detected is:
\begin{equation}
	I_\text{true} = \dfrac{I_\text{meas}-o}{g}.
\end{equation}
\subsection{Anscombe transformation}
\label{trafoAnscombe}
The Anscombe transform is used to transform a random variable with a Poisson
distribution into one with an approximatly constant standard deviation. The
transformation is defined as:
\begin{equation}
	A(x) = 2\sqrt{x+\frac{3}{8}}.
\end{equation}
As one can see in figure \ref{anscombe} the Anscombe transformations result has
for mean intensities greater than 4 a intensity independent standard deviation of
one.
\begin{figure}
	\centering
	\includegraphics[width = 0.5\textwidth]{pictures/anscombe.png}
	\caption{Standard deviation over mean intensities of different Poisson
	distributions}
	\label{anscombe}
	
\end{figure}

\section{Estimation of camera gain}
The 

\section{Distributions}
\subsection{Poisson distribution}
One very important probability distribution in physics is the Poisson
distribution. It describes the results of ``counting experiments'' and is
therefore very important for image processing as the pictures taken with a
camera are in principle counts of photons reaching the camera. Photon counting
noise is one important example.\\
Poisson distributions are just defined for integer values and the variance is
the same as the mean value of the distribution. Another important attribute is
the skewnes which is the inverse of the squarerot of the mean or variance and describes the assymetry.\\ 
The probability mass function is:
\begin{equation}
	p(n,\mu) = \frac{\mu^n}{n!}\exp(-\mu)
\end{equation}
\subsection{Skellam distribution}
The probability
mass function of a Skellam distribution is a function of the difference between
two Poisson random variables
\begin{equation}
	p(k;\mu_1, \mu_2) =
	\exp(-(\mu_1+\mu_2))\left(\frac{\mu_1}{\mu_2}\right)^{k/2}~I_{|k|}\left(2\sqrt{\mu_1
	\mu_2}\right)
\end{equation}  
where $n_1$, $n_2$ are the Poisson random variables and $k = n_1 - n_2$.
$I_{|k|}$ means the modified Bessel function of the first kind.\\
Mean $\mu$ and variance $\sigma$ of the Skellam distribution are given by
\begin{align}
	&&\mu &= \mu_1 - \mu_2,& \sigma^2 &= \mu_1 + \mu_2\\
	\Rightarrow &&\mu_1& = \frac{\mu + \sigma^2}{2},& \mu_2 &=\frac{-\mu +
	\sigma^2}{2}
\end{align} 

\subsection{Approach using skewness of poisson distribution}
For every pixel there is a set of multiple values in the set. This allows to
calculate the different parameters individually for each pixel. One can
calculate mean and variance of the measured intensities $I_\text{meas}(i,j)$ and
gets
\begin{align}
	\text{mean}(I_\text{meas}(i,j))& = \text{mean}(I_\text{true}(i,j)) + o\\
	\text{var}(I_\text{meas}(i,j))& = g\cdot\text{var}(I_\text{true}(i,j))\\
\end{align}
Assuming a Poisson distribution as the true intensity, mean and variance would
be the same. Unfortunately the mean true intensities are unknown and it is
not possible to determin $g$ and $o$ so far. For large mean Intensities $\mu$
the Poisson distribution becomes more and more similar to a Gauss distribution
with the same mean. However, for small means, the Poisson distribution is not
symmetric. The skewness $s_p$ of a Poission distribution is the inverse of the
square root of the mean $(\mu)^{-.5}$. It can also be directly
calculated from data
\begin{equation}
	s_p = \frac{1}{n}\sum_{i = 1}^n \left(\frac{x_i - \bar x}{\sigma}\right)^3
\end{equation}
The skewness is invariant to shift and multiplication with a constant. This
means that the transformation caused by the camera gain and the dark current
does not affect the skewness. This gives a third equation to solve for $g$ and
$o$.\\
This approach has very strict limitation to at least for background not to
intense signals. If the mean of the true Poisson distributin is higher than
roughly 30 the skewness gives due to noise no stabel results and it is
impossible to determin the mean intensity in this way.