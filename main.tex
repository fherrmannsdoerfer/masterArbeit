%\documentclass[a4paper,12pt,fleqn,bibliography=totoc,twoside]{scrreprt}
%fleqn 			--- macht die formeln links stat mittig
%bibtotoc		--- nimmt das literaturverzeichnis in das inhaltsverzeichnis auf

\documentclass[
  paper    = a4,
  BCOR     = 10mm,
  twoside,
  fontsize = 12pt,
  fleqn,
  toc      = bibnumbered,
  toc      = listofnumbered,
  numbers  = noendperiod,
  headings = normal,
  listof   = leveldown,
  version  = 3.03
] {scrreprt}
\usepackage{setspace}
\onehalfspacing
\usepackage[english]{babel} 
\usepackage[latin1]{inputenc}\usepackage[T1]{fontenc}


\usepackage{amsmath,amsthm,amssymb}

\usepackage{subfig}

\usepackage{array}
\usepackage{graphicx}%Bilder können eingebunden werden
\usepackage{subfig}

\usepackage{fancyhdr}
\pagestyle{fancy}
\fancyhf{}
\fancyhead[OR]{\rightmark} %die Section-Name
\fancyhead[EL]{\leftmark} % Chapter-Name
%\fancyhead[LE,RO]{\slshape \rightmark}
%\fancyhead[LO,RE]{\slshape \leftmark}
\fancyfoot[OR]{\thepage}
\fancyfoot[EL]{\thepage}
\def\MakeUppercase{}
%\geometry{bottom=42mm}
\usepackage{natbib}
\usepackage{caption}
%\captionsetup{format=plain}

\setcounter{tocdepth}{1}
\begin{document}
\include{titlepages-eng}
%% Abstract page
%% =============
%%
%% Content of abstract pages has been put into seperate pages to simplify
%% word counting. Use e.g. the unix command
%%   wc abstract-ger.tex
%% or
%%   wc abstract-eng.tex
%% to get the number of words contained in these files.
\thispagestyle{empty}
\begin{center}
  \begin{minipage}[c][0.48\textheight][b]{0.9\textwidth}
    \small
    \textbf{
      SimpleSTORM ein effizienter, sich selbst kalibrierender rekonstruktions Algorithmus f\"ur Einzel- und Mehrkanal-Lokalisationsmikroskopie:
    }\par
    \vspace{\baselineskip}
    Mikroskopie ist ein wichtiges Werkzeug der Zell-Biologie. Die H\"ochstaufl\"osende STORM Mikroskopie gew\"ahrt genauere Einblicke in Zellen als andere Arten der Lichtmikroskopie, \"uber die Abbe'sche Aufl\"osungsgrenze hinaus. Dies wird durch die Aufnahme vieler Bilder mit jeweils wenigen und damit leicht voneinander trennbaren Bildpunkten erreicht.\newline
SimpleSTORM ist ein effizienter Rekonstruktions-Algorithmus von STORM Daten, der von \cite{MAJoachim} entwickelt und im Rahmen dieser Masterarbeit weiterentwickelt wurde.\newline 
Für die Weiterentwicklung lag der Fokus, neben der Verbesserung des Algorithmus im Allgemeinen, auf der Vereinfachung der Bedienbarkeit im Besonderen. Durch Selbstkalibrierung ist die Angabe von Parametern wie dem Verst\"arkungsfaktor der Kamera oder der Breite der Punktantwort des Signals nicht mehr erforderlich. Diese werden automatisch bestimmt, k\"onnen aber auch vom Benutzer eingegeben werden. Dies erleichtert auch unerfahrenen Benutzern die Arbeit mit SimpleSTORM.\newline
Die Modellierung des Hintergrunds erlaubt es, Inhomogenit\"aten des Hintergrunds zu korrigieren und Aussagen \"uber die G\"ute der Signale zu treffen.\newline

Eine weitere Komponente der SimpleSTORM Software ist der Colorcomposer. Dieser erm\"oglicht die automatische Registrierung mehrerer Farbkan\"ale. Die Bestimmung der Kolokalisation zwischen verschiedenen Kan\"alen erlaubt die Untersuchung von Wechselwirkungen zwischen einzelnen Molek\"ulen.
  \end{minipage}\par
  \vfill
  \begin{minipage}[c][0.48\textheight][b]{0.9\textwidth}
    \small
    \textbf{
      SimpleSTORM an efficient selfcalibrating reconstruction algorithm for single and multi-channel localisation microscopy
:
    }\par
    \vspace{\baselineskip}
    Microscopy is an important tool in cell biology. Super resolution microscopy yields deeper insights into cells, further than Abbe's resolution limit. This is achieved by taking many pictures with few and therefore easy seperable fluorescent markers within each picture.\newline
SimpleSTORM is an efficient algorithm which was originally developed by \cite{MAJoachim}. The imporovements of the general performance and usability will be described in this thesis.\newline
Using selfcalibration it is no longer necessary to provide any parameter like the camera gain, camera offset or the expected signals point spread functions width. All crucial parameters are determined automatically but can also be manually adjusted. 
A model for the background is used to correct for inhomogenous backgrounds, and to determine the quality of the signals found.\newline

A second component is the Colorcomposer software. It provides easy auto alignment of different color-channels and measurments of colocalisation, which is an important measure to investigate molecular interactions.


  \end{minipage}
\end{center}

%\input{Titelseite}
\tableofcontents

\chapter{Introduction}
How does a virus reproduce? Which proteins are involved in chemical synapses? How do proteins interact? These questions and many more arise in biology and related sciences. Microscopy is a powerful tool to answer these questions, because it gives information about the distribution of proteins within a cell. However light microscopy is limited in spatial resolution due to limited diffraction, as described by \cite{Abbe} \footnote{There is a fundamental limit for the resolution of optical systems. It is roughly half the wavelength of the light which is used for microscopy. With green light of 500 nm this means the minimal resolution achievable is 250 nm}. Recently there have been developed different methodes to increase the resolution of light microscopy beyond the diffraction limit. To name a few: Photoactivated Localization Microscopy (PALM) \cite{Palm}, Stochastic Optical Reconstruction Microscopy (STORM) \cite{Storm} or Stimulated Emission Depletion STED \cite{sted}. For these techniques many images with sparsely distributed signals, coming from fluorophors which are blured by diffraction are used. Because the signals are sparse the center of each signal can be determined with sub pixel accuracy. This is not possible for one image with all signals, which would be blurred and undistinguishable. There are different ways to stain the biological sample with fluorophors. One way is to use antibodies which bind to specific targets within a cell. The antibody is either directly linked to a fluorophore or two antibodies are used, where the first binds to the biological structre and the second with the fluorophore attached to the first antibody. 
\newline
For this thesis only the STORM medode is considered. It can also be used to investigate the distribution of different proteins within a cell. To do so each protein is labeled with different fluorophores. Images can be aquired showing just signal from one kind of fluorophore. However, to be able to seperate the different signals from the flourophores, the emission spectrum must be distinct. This leads to cromatic aberration which results in images that are distorted and thus can not be aligned easily.\newline 
Many research groups have developed their own software to process STORM data sets. But most of these programs usually need many parameters, like the point spread functions width or the camera parameters, must be set by the user to get reasonable results, therefore these programs are difficult to use for somebody who is not familiar with image processing or does not know the parameters influence on the results.\newline

One key question of this thesis is how to build software that is easy to use even without prior information about the data or knowledge of image processing? \newline

Our answer to this question is SimpleStorm a software that calibrates itself. It estimates the camera parameters and the width of the point spread function of the fluorophores. In contrast to many other software applications, no threshold is needed.\newline
If more than one kind of fluorophore was used to stain the sample, the resulting images are distorted relative to each other. The colorcomposer tool was developed to do this task automatically and to analyse the aligned images. The colocalisation of the molecules of the different channels is a useful measure in biology to determine whether or not near molecules might interact. The colorcomposer software calculates both global and local measures of localisation.



\chapter{Theoretical background}
This chapter gives a brief overview over the most important distributions and transformations that are used for the SimpleSTORM algorithm. Also an introduction to the most important parts of a regular Charge-coupled Device (CCD) cameras is given. There is also a section about the data the algorithm is designed for.
\section{Distributions}
\subsection{Poisson distribution}
A very important probability distribution in physics is the Poisson
distribution. Its probability mass function is:
\begin{equation}
	p(n,\mu) = \frac{\mu^n}{n!}\exp(-\mu)
\end{equation}
The larger the mean value $\mu$ gets the more likely the Poisson distribution gets to a Gaussian distribution with a mean and a variance of $\mu$. Figure \ref{poisgaussdistr} shows Poisson and Gaussian distributions with different parameters. The mean value was shifted by 0.5 which results from continuity correction.\newline
Poisson distributions describe the results of ``counting experiments'' and are
therefore important for image processing as pictures taken with a
camera are in principle counts of photons reaching the camera.  \newline
Poisson distributions are defined for integer values only and its variance is
the same as the mean value of the distribution.\newline
The median of a Poisson distribution is approximatively given by
\begin{align}
	\text{median Pois}(\lambda) \approx \lambda + \frac{1}{3} - \frac{1}{50\lambda}
\end{align}
Another important attribute is
the skewnes which describes the assymetry.
\begin{align}
 \text{skewnes Pois}(\lambda) = \frac{1}{\sqrt{\lambda}}
\end{align}

\begin{figure}
\centering
\includegraphics[width = 0.9\textwidth]{pictures/poissgaussdistr.png}
	\caption{Different Poisson and Gaussian distrubutions. Especially for small numbers the Paisson and Gaussian distribution differ. The larger $\mu$ gets for the Poisson distribution the more similar it gets to a Gaussian with mean $\mu$ and sigma $\sqrt{\mu}$. The Poisson distributions were interpolated between their defined values.}
	\label{poisgaussdistr}
\end{figure}

\subsection{Skellam distribution}
The probability
mass function of the Skellam distribution is a function of the difference between
two Poisson random variables
\begin{equation}
	p(k;\mu_1, \mu_2) =
	\exp(-(\mu_1+\mu_2))\left(\frac{\mu_1}{\mu_2}\right)^{k/2}~I_{|k|}\left(2\sqrt{\mu_1
	\mu_2}\right)
\end{equation}  
$n_1$ and $n_2$ are the Poisson random variables and $k = n_1 - n_2$.
$I_{|k|}$ is the modified Bessel function of the first kind.\\
Mean $\mu$ and variance $\sigma$ of the Skellam distribution are given by
\begin{align}
	&&\mu &= \mu_1 - \mu_2,& \sigma^2 &= \mu_1 + \mu_2\\
	\Rightarrow &&\mu_1& = \frac{\mu + \sigma^2}{2},& \mu_2 &=\frac{-\mu +
	\sigma^2}{2}
\end{align}
Figure \ref{skellamdist} shows two Poisson distributions ($P_\text{red}$ and $P_\text{blue}$) and the resulting Skellam distribution (puple) of the difference $P_\text{blue} - P_\text{red}$.
\begin{figure}
\centering
\includegraphics[width = 0.7\textwidth]{pictures/skellamdist.png}
	\caption{Two Poisson distributions and the resulting Skellam distribution. The distributions were interpolated between the integer values.}
	\label{skellamdist}
\end{figure}



\section{The data}
\subsection{Labeling}
The concept of direct stochastic optical reconstruction microscopy (dSTORM) \cite{heilemann} is, to label structures of interest with fluorophors that can be exited using a laser with the appropriate wavelength directly, in contrast to STORM techniques which need an additional activator fluorophore. A short time after the activation the fluorophors emit a photon and fall back to the unexited state or lose the ability to get exited, they bleach out.\newline
The technique to stain the samples that was used for all real-world images in this thesis is called immunofluorescence. For this technique antibodies are used to attach the fluorescenic molecules to the samples. Antibodies target specific biomolecules within a cell that show their antigen. There are two classes of immunofluorescence techniques.\newline
Primary immunofluorescence uses only one antibody to that the fluorophore is directly attached.\newline
For secundary immunofluorescence two kinds of antibodies are used. One primary antibody that binds to the molecule or structure of interest and a second antibody that binds to the primary and carries the fluorophore.
\subsection{Description of the data sets}
The datasets for dSTORM microscopy that we recieve from our collaborators from
Bioquant are big datasets of several gigabyte in the Andor .sif format. Each
file conains a stack of pictures, normaly between 1000 and 10000, taken
consecutively with exposure times between 20 and 200 milliseconds.\newline

Figure \ref{rawStorm} shows a typical frame of raw data. In each frame there might be multiple fluorophores visible at the same time. Due to the large magnification, beyond the diffraction limit, the almost pointlike fluorophores appear as gaussian shaped signals, their point spread functions. The fluorophores are either attached to the biological structures that are of interest or they form a cluster, called a bead.\newline

Beads are larger and brighter than spots from only one fluorophore and are used to align multiple channels in the postprocessing step. Beads are designed to show up in every frame of the sequence at the same position and can therefore be used as landmarks for the alignment. They are composed of fluorophores of different colors to be visible in every channel.\newline

The other spots, bound to some proteins for example, only light up for a very short time. This is the key aspect of STORM. Instead of one frame that shows all fluorophores at the same time, thousands of frames are captured containing only a couple of point spread functions per frame. This gives the possibility to determine the center of each point spread function with a sub-pixel precision, and in the end when all points are displayed together in one picture, to an image with a resolution beyond the diffraction limit.

\begin{figure}
\centering
\includegraphics[width = 0.88\textwidth]{pictures/Pos2_2_red2-2frame2475Color.png}
	\caption{Raw image for dSTORM processing}
	\label{rawStorm}
\end{figure}

\section{CCD camera}
\subsection{Photon sources with shot noise}
The emission of photons is a random process that occures at unpredictible times. Therefore the number of photons passing through a plane is never constant but varies around some average value. The phenomena, that one can never determine exactly how many photons should hit the sensor chip of a CCD camera for example, is called shot noise. It playes a major role if the total number of photons is low, as from dark sources or with short exposure times of the camera.
\subsection{Quantum efficiency}
Quantum efficiency describes the fraction of photons that creates a detectable electron in a sensor chip. The quantum efficiency dependends on the wavelenght of the incoming photon. Photons with energies below the band gap can not produce a free electron that can be detected. The maximum of the quantum efficiency is basically caused by two effects. The higher the photons energy the higher the kinetic energy of the freed electron, but the electron is absorbed farer away from the detector and will therefore recombinate with a electron hole more likely.
\subsection{Gain}
There are two different gain factors involved in the capturing process of a camera. First the electric signal for each pixel might be amplified. Second there is a gain factor that describes the proportionality between collected electrons and the digital number it is associated with.
\subsection{Readout noise}
The origin of readout noise is the amplifier. The amplification is never perfect, this means that the exact number of electrons at the end of the amplification has some variation around the expected linearly increased value. There might also be some random signals of the electronics that add to the "true" signal. The readout noise does not depend on the exposure time.
\subsection{Dark current noise}
The dark current noise is generated by the thermal movement of the atoms in the sensor chip. The movement of molecules and atoms dependends on the temperature of the material, therefore dark current noise strongly depends on the temperature of the chip and can be reduced by cooling. Dark current noise generates electrons in the bins of each pixel. Even with closed shutter it is constantly increasing with time and follows Poisson statistics.
\subsection{Quantisation}
The signal must fit into the output color depth. It has to be rounded or truncated to fit in. This process introduces errors that can be seen as additional noise that is dependent on the intensity of the signal. High intensities are disturbed less relative to low intensies.


\section{Transformations}
\subsection{Transformation to Poisson distributed signal} \label{trafoPoiss}
The images aquired from the camera do not show the real intensities
$I_\text{true}$, which result from the photon emission of the probe, but
transformed ones $I_\text{meas}$. Two reasons why the taken image
differs from the true image are considerd.\\
There is dark current. This means that even a picture taken with closed shutter
would get some intensity, even without any light hitting the cameras sensor chip. This is the result of the thermal movement of the atoms in the sensor chip
and can be reduced by cooling. The dark current noise adds an almost constant
value $o$ to the output signal.
Incoming photons create electrons via inner photoelectric effect. This electrons
are collected for each pixel and might be amplified to get the final result.
Assuming a linear relation between the number of incoming photons and the number
of electrons created and a linear amplifier results in a factor $g$. This factor
is multiplied with the number of photons captured during exposure time for each
pixel.\\
If the gain factor $g$ and the offset $o$ are known the true intensity, the
number of photons detected is:
\begin{equation}
	I_\text{true} = \dfrac{I_\text{meas}-o}{g}. \label{transtopoiss}
\end{equation}
\subsection{Anscombe transformation}
\label{trafoAnscombe}
The Anscombe transform \cite{anscombe} is used to transform a random variable with a Poisson
distribution into one with an approximatly constant standard deviation. The
transformation is defined as:
\begin{equation}
	A(x) = 2\sqrt{x+\frac{3}{8}}.
\end{equation}
As one can see in figure \ref{anscombe} the Anscombe transformations result has
for mean intensities greater than 4 a intensity independent standard deviation of
one.
\begin{figure}
	\centering
	\includegraphics[width = 0.5\textwidth]{pictures/anscombe.png}
	\caption{Standard deviation over mean intensities of different Poisson
	distributions}
	\label{anscombe}
	
\end{figure}

\section{Estimation of camera gain}
\subsection{Using Skellam parameter} \label{skellam1}
The model assumes that the background and the beads are Poisson distributed. This data $I_\text{true}$ is then transformed in the following way:
\begin{equation}
	I_\text{meas} = g \cdot I_\text{true} + o \label{trafoGain}
\end{equation}
This is the inverse transformation of equatione \ref{transtopoiss}.\newline
Considering two pixels with different Poisson distributions $P_1$ and $P_2$ with mean value and variances of this distributions $\lambda_1$ and $\lambda_2$. If this distributions are transfomed as given in equation \ref{trafoGain} their mean and variance change like shown in equation \ref{meanvarPoiss1} and \ref{meanvarPoiss2}. This gives the oportunity to determine the gain and offset using two or more pixels.
\begin{align}
	\text{var}(I_{\text{meas}1})& = g^2\cdot\text{var}(I_{\text{true}1})\\ 
	\text{var}(I_{\text{meas}2})& = g^2\cdot\text{var}(I_{\text{true}2})\\
	\text{mean}(I_{\text{meas}1})& = g\cdot \text{mean}(I_{\text{true}1}) + o\\
	\text{mean}(I_{\text{meas}2})& = g\cdot \text{mean}(I_{\text{true}2}) + o
\end{align}
The values for $\text{var}(I_{\text{meas}1}), \text{var}(I_{\text{meas}2})$, $\text{mean}(I_{\text{true}1})$ and $\text{mean}(I_{\text{true}2})$ can be calculated from the data and can be used to calculate the gain as follows:
\begin{align}
	\frac{\text{var}(I_{\text{meas}1})-\text{var}(I_{\text{meas}2})}{\text{mean}({\text{meas}1})-\text{mean}(I_{\text{meas}2})}&= \frac{g^2\cdot \lambda_1  - g^2\cdot \lambda_2 }{g\cdot \lambda_1 + o - (g\cdot \lambda_2+o)}\\
	& = \frac{g^2\cdot(\lambda_1-\lambda_2)}{g\cdot (\lambda_1-\lambda_2)}\\
	& = g
\end{align}
The offset $o$ can be calculated in the same way.
\begin{align}
	\text{mean}(I_{\text{meas}1}) - \frac{\text{var}(I_{\text{meas}1})}{g} &= g\cdot \lambda_1 + o - \frac{g^2\cdot\lambda_1}{g}\\
	&= o
\end{align}
If more than two points are used a straight line must be fitted through them. For more robustness the Skellam distribution instead of the variances is used. It is favorable to perform this calculation on a wide range of intensities for better results.
\subsection{Using skewness of poisson distribution}
For every pixel there is a set of multiple values in the set. This allows to
calculate the different parameters individually for each pixel. One can
calculate mean and variance of the measured intensities $I_\text{meas}(i,j)$ and
gets
\begin{align}
	\text{mean}(I_\text{meas}(i,j))& = g\cdot \text{mean}(I_\text{true}(i,j)) + o \label{meanvarPoiss1}\\
	\text{var}(I_\text{meas}(i,j))& = g^2\cdot\text{var}(I_\text{true}(i,j)) \label{meanvarPoiss2}
\end{align}
Assuming a Poisson distribution as the true intensity, mean and variance would
be the same. Unfortunately the mean true intensities are unknown and it is
not possible to determin $g$ and $o$ so far. For large mean Intensities $\mu$
the Poisson distribution becomes more and more similar to a Gauss distribution
with the same mean. However, for small means, the Poisson distribution is not
symmetric. The skewness $s_p$ of a Poission distribution is the inverse of the
square root of the mean $(\mu)^{-.5}$. It can also be directly
calculated from data
\begin{equation}
	s_p = \frac{1}{n}\sum_{i = 1}^n \left(\frac{x_i - \bar x}{\sigma}\right)^3
\end{equation}
The skewness is invariant to shift and multiplication with a constant. This
means that the transformation caused by the camera gain and the dark current
does not affect the skewness. This gives a third equation to solve for $g$ and
$o$.\\
This approach is strictly limited to, at least for background pixels, low intensities. Tests have shown that if the mean of the true Poisson distributin is higher than roughly 30, the skewness gives, due to noise, no stabel results. For data sets with bright background intensities it is not possible to determine the camera parameters in this way.
%To estimate the camera gain and offset from the raw data the method described in \ref{skellam1} is used. Ideally the intensities for every pixel would follow a Poisson distribution with a certain mean. However this assumption only holds for beads and background pixel that never show any signal. The resulting points in the variance over mean plot will lie on a straight line. But what happens if background pixels are illuminated by a fluorophore in one or several frames?\newline
This section proves that points in the variance over mean intensity plot lie on the straight line or above, but almost never below.\newline
Set $A$ contains $n$ samples drawn from a Poisson distribution. The variance and mean of the set shall be $\mu_p$. Set $B$ is equal to set $A$ but the last ($n^\text{th}$) member $p_n^A$ is replaced by $p_n^A+c$. $c$ is a positive value that describes the difference in intensity between a background pixel and an illuminated pixel.\newline  

The exchange of the last pixel increased the mean intensity:
\begin{align}
\mu_B&=\frac{1}{n}\sum_{i=1}^n(p_i^B)\\
&=\frac{1}{n}\left(\sum_{i=1}^{n-1}(p_i^A) +p_n^A+c \right)\\
&=\frac{1}{n}\left(\sum_{i=1}^{n}(p_i^A) +c \right)\\
&=\mu_p + \frac{c}{n}
\end{align}
The variance of set $B$ times $(n-1)$ is given as:
\begin{align}
(n-1)\cdot\text{var}(B)&=\sum_{i=1}^n \left(p_i^B - \left(\mu_p+\frac{c}{n}\right)\right)^2\\
&=\sum_{i=1}^n \left(\left(p_i^B\right)^2 - 2p_i^B\mu_p-\frac{2p_i^Bc}{n}+\left(\mu_p+\frac{c}{n}\right)^2\right)\\
&=\sum_{i=1}^n \left(\left(p_i^B\right)^2 - 2p_i^B\mu_p+\mu_p^2 -\frac{2p_i^Bc}{n}+\frac{2c\mu_p}{n}+\frac{c^2}{n^2}\right)\\
&=\sum_{i=1}^{n-1} \Big(\underbrace{\left(p_i^A\right)^2}_{=p_i^B \text{for } i\neq n} - 2p_i^A\mu_p+\mu_p^2\Big)+\underbrace{\left(p_n^A+c\right)^2}_{=\left(p_n^B\right)^2} -2\left(p_n^A+c\right)\mu_p + \mu_p^2 \\
&~~+\sum_{i=1}^n\left(-\frac{2p_i^Bc}{n}+\frac{2c\mu_p}{n}+\frac{c^2}{n^2}\right) \nonumber\\
&=\sum_{i=1}^{n} \left(\left(p_i^A\right)^2 - 2p_i^A\mu_p+\mu_p^2\right)+2p_n^Ac+c^2-2c\mu_p\\
&~~+\sum_{i=1}^n\left(-\frac{2p_i^Bc}{n}+\frac{2c\mu_p}{n}+\frac{c^2}{n^2}\right) \nonumber\\
%&=(n-1)\mu_p+2p_n^Ac+c^2-2c\mu_p+\sum_{i=1}^n\left(-\frac{2p_i^Bc}{n}+\frac{2c\mu_p}{n}+\frac{c^2}{n^2}\right)\\
&=(n-1)\mu_p+2p_n^Ac+c^2+\frac{c^2}{n}-\sum_{i=1}^n\frac{2\left(p_i^B\right)c}{n}\\
&=(n-1)\mu_p+2p_n^Ac+c^2+\frac{c^2}{n}-\sum_{i=1}^{n-1}\frac{2\left(p_i^A\right)c}{n} -\frac{2\left(p_n^A+c\right)c}{n}\\
&=(n-1)\mu_p+2p_n^Ac+c^2+\frac{c^2}{n}-\sum_{i=1}^{n}\frac{2\left(p_i^A\right)c}{n} -\frac{2c^2}{n}\\
&=(n-1)\mu_p+2p_n^Ac+c^2+\frac{c^2}{n}-2c\mu_p -\frac{2c^2}{n}\\
&=(n-1)\mu_p+2c\left(p_n^A-\mu_p\right)+c^2 \left(1-\frac{1}{n}\right)
\end{align}
The exchange of $p_n^A$ increased the mean intensity, the variance of $B$ must exceed $\mu_p+1/n$ to lie above the line. Therefore the second part of the sum in equation \ref{gliwa} must be larger than $1/n$.
\begin{align}
\text{var}(B)&=\mu_p+\underbrace{\frac{2c\left(p_n^A-\mu_p\right)+c^2 \left(1-\frac{1}{n}\right)}{n-1}}_{>\frac{1}{n}}\label{gliwa}\\
&~~\frac{1}{n-1}\Big(2c\underbrace{\left(p_n^A-\mu_p\right)}_{> -\mu_p}+c^2 \left(1-\frac{1}{n}\right)\Big)&\overset{!}{>}\frac{1}{n}\\
\Rightarrow c &>\frac{\sqrt{\left(\mu_p^2+1\right)n^2-2n+1}+\mu_pn}{n-1}\\
&=\frac{\sqrt{\mu_p^2n^2+n^2-2n+1}+\mu_pn}{n-1}\\
&=\frac{\sqrt{\mu_p^2n^2+(n-1)^2}+\mu_pn}{n-1}\\
&\leq \frac{2\mu_p n}{n-1}+1 ,\text{because $\mu_p, n>0$ and $n>2$}\\
&\leq 2\mu_p + 1
\end{align}
This result confirms that if the additional intensity caused by signal $c$ is at least two times the mean intensity plus one of the background pixel, its variance rises.\newline
If n samples from a Poisson distribution with mean $\mu_p$ are drawn, and one sample has the lowest value possible, for a Poisson distribution, one, at least two times the mean value has to be added to increase the variance. Lower values for $c$ would decrease the summand $(p_n - \mu_p)^2$.\newline
The variance increses even stronger when the pixel is illuminated by a fluorophore more than once. The variance is not affected by a constant offset, but increases even more rapidly if a gain factor larger than one is present. In case of a gain factor $g$ the equations would get multiplied by $g^2$.
\chapter{Data processing}
STORM data from different cells or structures show many different features: there can be clusters of fluorophores with a high density of spots, areas with low density, variable background in space and time, beads can be present or absent.\newline
This section describes how the algorithm processes the data sets and how it is possible to find good settings that will work for all kind of input data.\newline
All real world images shown in this thesis were aquired from members of Prof. Dr. Mike Heilemanns group.


\section*{Import and processing}
STORM data usually has a size of around 3 gigabytes. However larger data sets are possible too, making it necessary to work on smaller parts of the data, instead putting the whole dataset into memory. This is done using chunks of a user defined size. The data is processed chunkwise, and the processing of the frames of each chunk can be parallelized. This parallelization is possible because the signals in each frame are considered to be independent from each other. There is a dependency between different chunks. The mean values of several chunks are used to estimate the background. Therefore a certain number of mean values has to be stored in memory.

\section{Workflow}
\subsection{Choosing parameters}
Initially the user has the option to set all important parameters. If no parameter is set the default ones are used and will give a good result because all crucial parameters are either determined from the data or set to reasonable values that work for every data set. The goal of the default parameters is to give a good result with no adjustment. This means parameters are chosen to produce almost 100 \% precision. It is assumed that the loss of some points that are not detected using this conservative setting will not affect the final result as much as a lower precision would.
\subsection{Estimating camera gain and offset}
First the application checks for a file containing settings for gain and offset from an earlier run. If this is not the case new parameters are estimated based on the first part of the data; usually 200 frames are sufficiant.\newline
A set of 2000 pixels is chosen by their mean intensity with respect to time. Therefore the mean intensity range is divided in 2000 bins. For each bin one pixel with the appropriate mean is selected. For faster computation the mean is not calculated over all frames but only over a certain, user defined, number. The goal is to get pixels with the whole range of mean intensities, good representatives from the darkest background pixel to the brightest beads.\newline 
The method described in section \ref{skellam1} is used to estimate the gain factor, based on the preselected points. The variances of the pixels are computed based on the first part of the data.
Each dataset is three-dimensional, where time is the third
dimension. Therefore mean $\mu$ and variance $\sigma^2$, in time, can be calculated from
the data for each pixel individually
\begin{align}
	\mu(i,j) & = \frac{\Sigma_t(I_t(i,j)(i,j))}{n}\\
	\sigma^2 & = \frac{\Sigma_t(\mu(i,j)-(I_t(i,j)))^2}{n-1}\\
\end{align} 
$I_t(i,j)$ describes the intensity of the pixel of frame $t$ at position $i,j$.
To determine the gain factor the variances $\sigma^2$ for each pixel are plotted over the mean
intensities. A straight line can be fitted, and its slope gives the gain
factor, the $x$-intersection gives the offset.\newline 
Figure \ref{skellamplot} shows the scatter plot for the selected points and the fitted line. The data is taken from a real world data set. The red dots result from pixels that have a constant mean intensity over time. They are either dark the whole time, these are the dots on the left or they are beads and show signal in every frame. The blue dots result from pixels that are sometimes covered by a PSF. This blinking has much stronger influence on the variance than for the pixels mean values. This is the reason why this points are clustered on the lower end of the mean intensities. \newline
For a robust estimation of the straight line a RANSAC (random sample consensus) algorithm is used. It is described in section \ref{ransacdescr}. 

\begin{figure}
\centering
\includegraphics[width = 0.99\textwidth]{pictures/skellamplot.png}
	\caption{Scatter plot for the preselected points. Blue points result from pixels that show at least once a higher intensity caused by a fluorophore. The red dots are used to determine the gain and offset.}
	\label{skellamplot}
\end{figure}
\subsection{Recursively adjusting gain and offset}
After the estimation of gain factor and offset, the transformations described in \ref{trafoPoiss} and \ref{trafoAnscombe} are applied and the background is subtracted.\newline
Due to the Anscombe transformation, the background pixels of the image should only vary around a mean intensity of zero with a variance of 1. Therefore, a histogram of the pixel intensities is created. After background subtraction, the background pixels should contribute only to the lower intensities in the histogram. A Gaussian function is fitted to the histogram's values. This is done under the assumptions that there is much more background in the image than signal, or that the intensities coming from signals are distributed over a larger range, so the Gaussian for the background intensity distribution can be fitted correctly.\newline


If the estimated value for the variance is too far from 1, the originally estimated gain factor is corrected and applied, and the fit is done again (compare section \ref{checkGain}). This is done until the background variance converges within a small threshold or the maximal number of iterations is reached. In this case the initial gain factor will be used and a warning will be shown on the screen.\newline
It might be that the correction of the gain factor gets trapped duo to overshoting. In that case the estimated background variances oscillate around one. If this is the case the mean gain factor from the last two corrections is used.
\subsection{Estimating the width of the point spread function}
For 20 local maxima of a certain, user defined, number of frames the square of the Fourier transformation off each roi around the chosen local maxima is calculated. This squared Fourier transformations are then averaged. The result is called the mean power spectrum. It can be used to estimate the variance of the point spread function of the signal. A two dimensional Gaussian function's Fourier transform is again a Gaussian but with inverse variance. This relation is used to determine the variance of the point spread function in the spatial domain, using the fit parameter for the variance in the frequency domain.\newline
The script for fitting the two dimensional Gaussian function was implemented by Ilia Kats.
\subsection{Processing the data}
\subsubsection{Import Data}
Storm data sets can consist of several thousand frames with resolutions up to one mega pixel per frame. This size makes it necessary to break the data into smaller parts, otherwise it could be much larger than the RAM of an ordinary machine. Because of the background estimation, it is not possible to process every frame completly independently as it was in the older version of this software (\cite{MAJoachim}). 
%For the hdf5 data format it is faster to load a larger consecutive part of the dataset into memory insetead of loading each frame.
\newline
This algorithm uses chunks of user defined size. There are some limitations to the chunk size that are discussed later. The data set is split into parts of equal size in the $x$- and $y$-dimensions and independently also in the $t$-dimension. If this partition does not fit at the edge of the data set, the last chunks will be smaller.\newline
The data is transformed to be Poission distributed, then the Anscombe transform is applied which gives background intensities with unit variance.\newline
The implementation of the workflow using chunks was done by Ilia Kats.
\subsubsection{Background estimation} \label{bgestimation}
For each chunk, the median of the Anscombe transformed data is determined to get a robust estimate of the background value for this chunk. 
B-spline interpolation, implemented in vigra (\cite{vigra}), is used to get interpolated values for the full resolution of the current frames. For this interpolation, three chunks in both the spatial and the temporal domain have to be available. Therefore the maximal chunk size into $t$-dimension must not be larger than a third of the total stack size, the same holds for the chunk size in $x$ and $y$ direction which must not exceed the spatial resolution of the image stack. These values are checked automatically and changed if necessary.\newline
The interpolated background is then subtracted from the transformed data to give background pixels with zero mean and unit variance, both in $xy$- and in $t$-dimension. Figure \ref{removedBG} shows an frame with variable background before and after background subtraction.
\begin{figure}
\subfloat[Original image]{\includegraphics[width = 0.485\textwidth]{pictures/Tubulin2OrigFrame50Color.png}}\hfill
\subfloat[image after background subtraction]{\includegraphics[width = 0.485\textwidth]{pictures/Tubulin2normalprocessFrame50Color.png}}
	\caption{Effect of background subtraction on an inhomogenous background. For the human eye no more points are visible but for computers it is much easier to find the bright spots in the background subtracted image, because a global threshold can be applied or as in the case of SimpleSTORM the probabilistic model can be tested.}
	\label{removedBG}	
\end{figure}
\subsubsection{Create mask for background suppression}
With the given $p$-value from the settings, a global threshold can be determined, because inhomogenities of background intensities have been removed. The threshold value is that intensity 
for which the integral over the probability density function of a Gaussian distribtution with zero mean and unit variance gives $1-p$. It is that intensity for which it is less likely than $p$ to occure. This threshold is possible because the background intensities follow a Gaussian distribution, with known mean and variance, after all transformations applied.\newline
The threshold is applied to the current frame and stored as a mask. For the reason that it is possible for bright background pixels to exceed the threshold, the connected components of the mask are calculated. Pixels that belong to connected components with too few members are discarded. The idea behind using connected components is, that there is a probability for a background pixel to be brighter than the threshold, but it is unlikely that two neighbouring pixels exceed the threshold in the same frame and even more unlikely that three of them do so. Therefore the number of pixels necessary for keeping a connected component is set to a minimum of three. If the width of the point spread function is very large the number of pixels can be set depending on the PSF width. But to suppress background pixels efficiently the number of pixels in a connected component has to be larger than two.
\subsubsection{Filtering data and finding maxima}
To improve the accuracy of the spot detection, the transformed signal is convolved with a two dimensional Gaussian function with the previously determined or user-set width. The convolved image will further be used to find the maxima. All local maxima in the frame are detected. Each maximum found is tested to be covered by the mask or discarded otherwise. A region of interest around the remaining maxima is interpolated to a higher resolution. The interpolated region is searched for local maxima once again. These maxima will be detected with super resolution.\\
To determine the signal-to-noise ratio, the unfilterd and uninterpolated pixel intensity is used.

\subsubsection{Quality control for detections}
Sometimes, especially in data sets with a high density of spots, two spots are near enough that their point spread functions overlap. It may happen that instead of two maxima only one maximum will be detected between the true ones, as can be seen in \ref{betterthansimplestorm}. This leads to large errors in the localization. To avoid this a threshold for the asymmetry of the spots can be set.\newline
The calculation of the asymmetry was already implemented by Joachim Schleicher \cite{MAJoachim}.


\section{Comparison with older version of the SimpleStorm algorithm}
\subsection{Adjustable filter width} \label{sectionFilterisEvil}
If two or more point spread functions (PSF) overlap, applying a smoothing filter can lead to the merging of point spread functions. This merging becomes a problem if the two distinct maxima of the original unsmoothed PSFs form a new maxima in between. This leads to just one detection somewhere between the true maxima. The number of merging psfs increases with greater filter widths. For high density data it might be better to use a filter with smaller width and therefore less accuracy, but fewer merged PSFs. In total this might give a better result depending on the number of incorrectly merged PSFs. This is the reason why filtering was changed to use just a Gaussian filter instead of the Wiener filter originally used. 
The effect of the smaller filter width can be seen in Figure \ref{betterthansimplestorm}. The red crosses indicate detections found by the new SimpleStorm version, the green crosses show the estimated positions found by the previous version of SimpleStorm, and the white x marks the true location. The predictions by the newer version are not perfect but better than the predictions of the older version. The scores that can be seen in Table \ref{tabelbetterthansimplestorm} are calculated like as described in section \ref{measuresISBI}. Both results have almost the same accuracy, but differ in the scores. There are two effects related to the accuracy canceling each other out. Fewer merged PSFs increase the accuracy, but the positive effect of filtering with the appropriate filter, as shown in \ref{accplot2} or in \ref{matchedFilter1}, is lost.

\begin{table}
\caption{Comparison of results created using almost no smoothing or Wiener filter on high density data. Although the accuracy is almost the same, the number of detections and the scores are higher for the unsmoothed data. For the evaluation software from the \cite{challenge} were used.}
\begin{tabular}{l|llllll}
&intersections&Jaccard&F-Score&Precision&Recall&RMSE\\ \hline
Gaussian filter, width = 0.01& 16955&19.99&33.26&81.10&20.92&27.21\\
Wiener filter& 14480&17.06&29.15&79.19&17.87&27.28
\end{tabular} \label{tabelbetterthansimplestorm}

\end{table}


\begin{figure}
\subfloat[Advanced setting widget]{\includegraphics[width = 0.485\textwidth]{pictures/betterthanSimplestorm1improved.png}}\hfill
\subfloat[Easy setting widget]{\includegraphics[width = 0.485\textwidth]{pictures/betterthanSimplestorm2improved.png}}
	\caption{This pictures show the effect of a smaller filter width. The red crosses show detections found with a filter width of 0.01, the green crosses the results using a Wiener filter. The white x marks the ground truth.}
	\label{betterthansimplestorm}	
\end{figure}

\subsection{False positive suppression}
In the previous version the background was determined by first estimating a baseline. The minimum of the current frame was taken as the baseline. It was subtracted from the image and the resul was smoothed with a Gaussian filter of widht 10. The smoothed image was subtracted from the original image to give a background free image. This works fine for subtracting background with variations larger than the filters width, but the resulting intensities do not contain any information about the variability of the background intensities.\newline
If, for example, the resulting intensity, after background subtraction, is 5. This can either mean it is most likely signal, given that the variance of the background in the original image was very small, or it can mean nothing, if the variance of the original background was 20. In the latter case the probability that the difference of 5 between the original image and the smoothed one has a high chance to result from the background variation.\newline
In the older version of SimpleSTORM the intensity of a candidate, considered to be a signal, was checked by comparing its intensity after the background subtraction with its intensity before. If the intensity of the maximum was at least twice as high as the subtracted background, it was taken for further processing, otherwise discarded. If the background had a high variability in the spatial domain, the baseline got a value that was too low for regions with a higher mean intensity. For this regions the background that was subtracted had high values and therefore the maxima were discarded because the resulting intensity was less than twice the background. This can be seen in Figure \ref{bgmakesitbad}. The old version of SimpleSTORM only findes about 8000 spots while the new version finds more than 44000 on a test data set with high background variance. Using the new background suppression results in a reconstructed image that shows the reconstructed structures in a better way.

\begin{figure}
\subfloat[Typical frame showing variable background intensities]{\includegraphics[width = 0.3\textwidth]{pictures/Tubulin2OrigFrame50.png}}\hfill
\subfloat[Result old SimpleSTORM]{\includegraphics[width = 0.30\textwidth]{pictures/Tubulin2factor1OldSimpleSTORM.png}}\hfill	
\subfloat[Result new SimpleSTORM]{\includegraphics[width = 0.3\textwidth]{pictures/Tubulin2factor1.png}}

\caption{This pictures show the drawback of the old background treatment. On the left a typical frame with variable background is shown. The old version of SimpleSTORM discards many detections in the regions of high background intensity. The new SimpleSTORM software discards detections based on their signal-to-noise ratio and is less affected by variable background.}
\label{bgmakesitbad}	

\end{figure}

The great advantage of the newer version of SimpleStorm is its model for the background. For each pixel, the probability to be signal can be determined based on its signal to noise ratio (SNR). Using the number of connected components of each cluster, false positive detections caused by bright background pixels can be suppressed.

\subsection{Comparable results based on the signal-to-noise ratio}
The accuracy of the maxima detection relies on the signal-to-noise ratio. With a given SNR the correct standard deviation of the localization error can be used for further calculations. This is not possible if only intensities are saved because without information about the backgrounds variance the reliability of the detection can't be estimated.\newline
Saving the signal-to-noise ratio also enables to compare results from either different cameras or different settings or a different environment that might lead to a higher background variability.  


\section{New graphical user interface (GUI)}
\subsection{Input widget}
The new GUI for SimpleStorm was designed to integrate its many new features. Figure \ref{guiWidgets} shows this new design. \newline
There are three categories of parameters. The first category specifies which upsampling factor will be used, the width of the pixels in nanometers of the input data, the reconstruction resolution, the number of frames that are used to estimate the camera parameters and the PSFs width and sensitivity of the algorithm. The first three parameters are general informations that depend on the capturing process of the data and the desired upsampling factor. Two of them must be set and the third is calculated automatically.\newline
The most challenging parameters of this section are the alpha value, which sets the sensitivity for false detections and the number of frames used for estimation of the camera parameters and the PSF. For these values the default setting are set to work well with any data sets. \newline
The next category of parameters defines the width of the region of interest (roi) for the estimation of the point spread function and the chunk sizes in spatial and temporal dimension can be set. There are two different ways to set these parameters (see Figure \ref{guiSettings}). One is to give values for all parameters. This is difficult without understanding the influence of these parameters on the algorithm. Therefore there is also a second way to set this parameters. The user should know some properties of the data that shall be processed, such as: is the spot density high or low? Is there variable background in time and space? Depending on the sliders' positions, the best parameters are set automatically. How this is done will be described in section \ref{easyParam}. With this second option the user can process his or her data, treating variable background or dense data without deep insight or understanding of the algorithms.\newline
The last catagory of parameters describes the camera gain and offset, the width of the signals point spread function, and a prefactor that can be used to alter the estimated gain. 
\begin{figure}
\subfloat[Easy setting widget]{\includegraphics[width = 0.485\textwidth]{pictures/basicSettingsGui1.png}}
\subfloat[Advanced setting widget]{\includegraphics[width = 0.485\textwidth]{pictures/advancedSettingsGui.png}}\hfill
	\caption{There are two different ways to set the parameters for the algorithm. On the right the standard way of setting parameters can be seen. On the left, there are sliders that can be adjusted between the extremes. The program sets the value for the parameters in a way to produce the best results for the selected attributes of the data set.}
	\label{guiSettings}	
\end{figure}
\subsection{Result widget}
After the run button at the lower left edge of the input widget is pressed, a new tab opens. In this widget the reconstructed image is shown. At the bottom there is a progress bar that displays the current processing step and its progress. Buttons to zoom in and out or to fit the displayed image into the window are located in the lower part of the result widget. On the lower right there is the stop/save button. It either stops the program if it is still runing or opens a dialog to save the result image and the coordinates of the detection if the program has already been stopped.\newline
The GUI was mainly designed by Ilia Kats.

\begin{figure}
\subfloat[Input widget]{\includegraphics[width = 0.405\textwidth]{pictures/InputWidget.png}}\hfill
\subfloat[Result widget]{\includegraphics[width = 0.575\textwidth]{pictures/ResultWidget.png}}
	\caption{The new GUI. On the left is the window for selecting input file and parameters. On the right is the result widget showing the processing of a data set in progress.}
	\label{guiWidgets}	
\end{figure}

\subsection{Easy parameter selection}\label{easyParam}
As mentioned above the easy setting widget makes it easy to set reasonable parameters without knowing their influence on the algorithms.\newline
The background sliders have direct influence of the corresponding chunk size. A more constant background gives better results with larger chunks. This is because the median of all pixels in a chunk is used to estimate the mean value of the background. For the estimation of this relation between the median and the mean $\lambda$ of a Poisson distribution, described in \ref{meanMedianPoiss}, is used.
Because of the high mean values for $\lambda$ the last summand can be dropped.
This estimation works only if the chunks contain more background pixels than signal, otherwise the mean value will be overestimated. With large chunks this assumption is satisfied. But the smaller the chunk size, the more likely it is that a cluster of points lies in the chunk, and the values of the median are too high. On the other hand, the chunk sizes should be within the range of the changes in background. The best chunksize is therefore in the range of the variable background. The slider position sets the chunk sizes linearly to a value that lies in between the smallest and the largest chunk size possible for the data set.\newline
The minimal possible chunk size is 3 pixels. The largest chunk size possible is half of the shorter border in $x$ and $y$ dimension and the number of of frames over which the parameters are estimated divided by the number of chunks in memory in $t$ dimension. These restricitons to the maximal chunk size result from the spline interpolation as described in \ref{bgestimation}.\newline 
In general, the denser the dataset is, the more likely it is that two neighbouring PSFs are merged due to the filtering, as shown in \ref{sectionFilterisEvil}. There are two ways to avoid inaccurate detections that result from merged PSFs. One is less filtering, to avoid merges, the second one is checking the symmetry of the detected spots. Merged spots become more and more asymmetric the further the two true centers of the PSFs lie apart from each other. A high asymmetry is a good indicator for a detected spot with low accuracy.
The spot density influences the prefactor for the estimated sigma and the value for the asymmetry checks. It sets the threshold for the asymmetry in the same way the sliders for background work, within appropriate limits, with higher values for the asymmetry threshold for less dense data sets. Also, the prefactor is set to values between 1 for sparse data and zero for dense data.
\input{CheckForAssumptions}
\chapter{Multicolor registration}
In microscopy it is often desirable to label different structures in a cell with
different colors. To do so our collaborators use different fluoroscent molecules
that emit light at different and distinguishable wavelengths. Using different
filters it is possible to capture pictures just containing light emited from one
fluorophore. To get a mulit-channel picture the different channels must be
aligned. Because different flourophores emit different wavelengths, chromatic
aberration apears. This means that the light for the same spot but with
different wavelenghts is not mapped to the same spot in the image. To align the
different channels despite chromatic aberration, beads are used. Beads are
flourophores added to the probe that emit light in all wavelengths the
different markers do, and therefore are visible in all channels. The beads can be
used as landmarks, because their position in the original image is at the same
spot. The task is to find a transformation that maps corresponding beads on each
other.

\section{Colorcomposer GUI}
The goal for the colorcomposer tool is to provide software that is easy to use, flexible and powerful. The current version of the colorcomposer is easy to use, because the different channels can be aligned by just selecting the auto align option.\newline
But it is also flexible. If the user wants to select the beads on his or her own, the beads can manually selected and deleted. After the transformation the user might use the implemented tools for colocalisation detection or save the transformed images and process them with the tool of his or her choice.\newline
Also the colorcomposer is powerful as it provides for example information about the number of points currently under the cursor or its intensity. Also the estimated transformation error and the localisation error are computed and stored.\newline
The basic framework for the colorcomposer was set up by \cite{MAJoachim}. It contained the workflow for importing and exporting images the handling of bead objects and a linear transformation that used the beads in the order they were found. This early version was not usable and were improved.\newline
Figure \ref{ColorComposer} shows improved colorcomposer GUI with two datasets loaded. The buttons on the right give the user the option to add or remove beads also in addition to the autodetected beads. There are also different sliders to control the values used for bead detection. In the lower right corner there is additional information provided about the total number of points within a rectengular with the selected cursor radius' size, also the sum of the intensities and the total number of frames for each data set is given. This information are helpful to determin whether or not a cluster of points is a bead or not.\newline
On top there is a menu bar with new options, like discarding all beads, automatically detecting beads, calculate colocalisation measures and show or hide the colocalisation heatmap.
\begin{figure}
\centering
\includegraphics[width = 0.88\textwidth]{pictures/GuiColorcomposer/BeforeAlignmentWithMenu.png}
\caption{Improved colorcomposer GUI. On the right sliders to set the parameters are present. There are also buttons to add or delete singel beads. On the lower right additional information about the area under the coursor is displayed, such as number of pixels, total intensity and the number of frames in total.}
\label{ColorComposer}
\end{figure}


\section{Features of the colorcomposer application}
\subsection{Invariance of input datas units}
The resulting coordinate files from SimpleSTORM or other STORM algorithms may be given in units of pixels relative to the unprocessed data or in nanometers. Treating coordinates given in nanometer as pixlel units would lead to very huge and very sparse images to display in the colorcomposer. Therefore the colorcomposer reads out additional information from the coordinates text files header. This informations are the pixel to nanometer ratio and the used factor. With this information the picture can be reconstructed as an upsampled version of the input image by the used factor regardless of the units used to save the coordinates file. If none of this information is given a pixel to nanometer ratio of 1 is assumed which garantues backward compatibility with older coordinate files given in pixel units.
\subsection{Manual bead selection and removal}
With the improved version of the colorcomposer it is possible to add beads manually. Therefore the desired location is clicked in the preview image and after that either the button "add green bead" or "add red bead" is hit. If there are enough frames containing localisations near the given location a bead is added to the center of mass of the intensities in that area. If the button "delete bead" is pressed all beads in the selected area are deleted.\newline
This feature can be used to add beads which the automatic bead detection missed.
\subsection{Automatic bead detection}
The input for the colorcomposer application is a text file created by the storm
algorithm that contains information about the position, intensity, symmetry,
frame number and signal-to-noise ratio of each detection. The beads should
ideally appear in most of the images. This means they can be found by searching for
detections that appear in almost every frame at the same position.\newline
There was already an automatic bead detection implemented by Joachim Schleicher. This was improved in the following ways.\newline
All important parameters for the bead detection can now be set in the GUI. The bead detection works by searching for points that appear in most of the frames. Instead of taking all localisations from the first frame as expected bead positions without considering locations that does not appear in the first frame, now a good subset of positions from the first 50 frames by skipping redundant positions based on the minimal distance of a new position to all positions already in the set. The range of 50 frames to look for beads is sufficiant because it is very unlikely that all 50 detections of the bead have been missed.\newline
After good candidates are found their number of points, variance and mean position is determined like described by \cite{MAJoachim}.\newline
In the end beads that are too close together are merged to for a new bead with its center right between the merged beads.

\subsection{Alignment of two multicolor images}
After the beads for each channel are found, the next task is to find the corresponding
beads in each channel. It can happen that some beads occur in just one channel.
If this is the case there will be no corresponding bead in the other
channels.\newline
To align the beads, the minimal number of beads, three, that are neccesary to
calculate the transformation are chosen randomly from the first channel. After that, based on
a probabilistic approach and a distance matrix containing information about the
distances between all beads of the two channels, three beads from the
second channel are chosen. It is more likely for nearer beads of the other channel to be selected, but any bead within a certain range can be chosen.\newline
Using these pairs of beads, a linear transformation is found, as described by
\cite{MAJoachim}.\newline
This transformation is used to tested how many other beads, that were not used to calculate the transformation match intotal. It is assumed that the correct transformation will match other bead pairs in addition.
This is very important because with every set of three points a valid transformation can be found that perfectly aligns this three beads. After that the whole
procedure is done multiple times. In the end the best transformation is
chosen based on the total number of bead pairs that match. If there are multiple transformations that match the same number of bead pairs, it is searched for the transformation with the lowest root mean square error for the matching beads.\\
In principle shearing should also be allowed for a linear transformation, but tests
indicate that shearing does not occur, so it is disabled to improve stability. If there are just
three beads in each channel, then every time a perfect transformation is found,
but with the constraint of forbidden shearing, the right solution can be
identified. There is an other problem with this transformation if the bead density is very high it might be that a transformation with much shearing is found that compresses the beads of one channel to a slim band. The probability to find a matching point by chance then is much greater then. Figure \ref{badshearing} shows the result of an incorrect transformation of simulated data. The red channel was created by randomly placing beads. The green channel was slightly shifted and rotated. The green channel was transformed to match with the red channel. Just a subset of beads were used to calculate the transformation and so this solution was found and chosen from the algorithm because of the additional matching point.\newline
This effects can be suppressed by not allowing shearing for the transformation.

\begin{figure}
\centering
\includegraphics[width = 0.88\textwidth]{pictures/shearingBad.png}
\caption{Affine transformation with shearing enabled. The green channel is deformed and one bead matches by chance which led to selection of this transformation.}
\label{badshearing}
\end{figure}
\subsection{Information about localisation certainty}
The detected spots from SimpleSTORM contain some localisation error. This error will be derived in section ref{detectionError}. It depends on the signal-to-noise ratio and the scale of the point spread function of a single fluorophore.

\section{Total localisation error}
There are four contributions to the total localisation error considered in this thesis.\newline
First there is the error introduced from the linear transformation to align the beads. To estimate this error the variance of the estimator is used. Since the transformation matrix is calculated by linear regression from the matricies $B$ and $R$ of the localisations of the first and the second layer of beads, with $B$ beeing the matrix of the first layer to that the second layer $R$ is transformed to. The variance of the estimator is
\begin{align}
\text{variance registration}&= \sigma^2 x_0\left(R^TR\right)^{-1}x_0^T, \text{ with }x_0 = \left(x_o,x_1,1\right) \label{gl22}
\end{align}
Using equation \ref{gl22} the variance for all pixel can be calculated.\newline
The second contribution to the total localisation error is the localisation error of the SimpleSTORM algorithm. Its formula is
\begin{align}
 \text{variance localisation} = \frac{N^2\pi}{2S_0^2} \left(1+\frac{\sigma_\text{PSF}^2}{\sigma_\text{filter}^2}\right)^2\left(\sigma_\text{filter}^2+\sigma_\text{PSF}^2\right)^2
\end{align}
It's derived in \ref{detectionError} and can be calculated for each detection individually based on the known signal-to-noise, which gives directly the signals intensity $S$ because of the known noise variance of one which gives the nois' standard deviation $N$ of also one. The PSFs width was either given or estimated and is passed to the colorcomposer in the detection coordinate file's header along with the prefactor $f$ for the actually used filter. Using this the localisation error can be writen as
\begin{align}
 \text{variance localisation} &= \frac{N^2\pi}{2S_0^2} \left(1+\frac{\sigma_\text{PSF}^2}{f^2\sigma_\text{PSF}^2}\right)^2\left(f^2\sigma_\text{PSF}^2+\sigma_\text{PSF}^2\right)^2 \\
 &=\frac{N^2\pi}{2S_0^2}\left(1+\frac{1}{f^2}\right)^2\left(1+f^2\right)^2\sigma_\text{PSF}^4\\
 &= \frac{N^2\pi}{2S_0^2}\left(2+\frac{1}{f^2}+f^2\right)^2\sigma_\text{PSF}^4
\end{align}
The third contribution results from the fact that the fluorophors are not directly attached to the sample of interest, but there are the antibodies used for staining inbetween. The upper bound of this error can be estimated under the assumption that the target of the antibodies is much smaller than the antibodies, so that there is no occlusion in the projection. This assumption does not hold in reality. Structures like cell membranes are certainly larger than the antibodies, but the assumption gives an upper bound. 
\begin{align}
\text{var x} &=\frac{1}{\Omega_\text{Sphere}}\int\limits_{\Omega_\text{Sphere}} x^2 dV\\ \frac{1}{\Omega_\text{Sphere}}\int\limits_0^{2\pi}\int\limits_0^\pi R^2\sin\theta \left(R\cos\phi \sin\theta\right)^2 d\theta d\phi\\
&=\frac{R^2}{4\pi} \int\limits_0^{2\pi}\int\limits_0^\pi \cos^2\phi \sin^3\theta d\theta d\phi\\
&=\frac{R^2}{4\pi} \left[\frac{1}{2}\phi +\frac{1}{4}\sin 2\phi \right]_0^{2\pi}\int\limits_0^\pi \sin^3\theta d\theta \\
&=\frac{ R^2}{2} \left[\frac{1}{12}\cos3\theta -\frac{3}{4}\cos\theta \right]_0^\pi\\
&=\frac{2}{3} R^2
\end{align}
$R$ is the assumed distance of the fluorophore to the structure it is attached to.\newline
The fourth and minor contribution to the total localisation error for each pixel is the quantization noise. It occurs when continous intensities, as the photons forming the PSF, are transformed into integer values. Quantisation noise describes the round-off error. The round-off error can be any value between -0.5 and 0.5 and is uniformly distributed. Its variance $\sigma_Q^2$ is
\begin{align}
 \sigma_Q^2 =\int\limits_{-\frac{1}{2}}^{\frac{1}{2}} x^2 dx = \frac{1}{12}
\end{align}
The unit of this error is the pixel size of the upscaled image and therefore more and more negligible the higher the upscaling factor is.\newline
To get the total localisation error all four variances are summed up for each pixel.

\section{Colocalisation}
Colocalisation in wide field microscopy is a measure of the overlap of data point from different channels. It can provide information whether or not two molecules interact. With increasing resolution of the images colocalisation becomes more and more a measure of similar structures near each other. Two structures can't be at the very same position in the cell. The colorcomposer software provides both global and local colocalisation measurements.
\subsection{Global colocalisation}
The most common colocalisation measure is Pearsons correlation coefficient \cite{pearson}. It is given as:
\begin{align}
\text{Pearson correlation coefficient =}\frac{\sum ^n _{i=1}(X_i - \bar{X})(Y_i - \bar{Y})}{\sqrt{\sum ^n _{i=1}(X_i - \bar{X})^2} \sqrt{\sum ^n _{i=1}(Y_i - \bar{Y})^2}}
\end{align}
It is the ratio between the covariance between the points of two channels and their standard deviation.\newline

Also the Manders correlation coefficients $M_1$ and $M_2$ and the overlap coefficient (\cite{manders}) are calculated.
\begin{align}
M_1 =& \frac{\sum_i R_{i,\text{coloc}}}{\sum_i R_i}&M_2 = & \frac{\sum_i G_{i,\text{coloc}}}{\sum_i G_i}
\end{align}
With $R_{i,\text{coloc}} = R_i$ if $G_i >0$ and $R_{i,\text{coloc}} = 0$ otherwise and $G_{i,\text{coloc}} = G_i$ if $R_i >0$ and $G_{i,\text{coloc}} = 0$. $G_i$, $R_i$ are the intensities of the pixel of the green and red channel.
\begin{align}
\text{overlap coefficient} = \frac{\sum_i R_i \cdot G_i}{\sqrt{\sum_i \left(R_i\right)^2 \cdot \sum_i \left(G_i\right)^2}}
\end{align}

\subsection{Local colocalisation}
Global colocalisation has the drawback that there is just one value for the whole image. If there are regions in the image that show much colocalisation and other regions without colocalisation the same colocalisation coefficient might be achieved as if one channel is distributed randomly.\newline
For local colocalisation analysis the algorithm from \cite{coloc} are used and were further developed to gain a speed boost. The algorithm runs now approximatly 40 times faster. This was achieved by using scipys (\cite{scipy}) ckdtree function, which is a k-d tree implemented in C.
%\subsection{Validation of colocalisation approaches}
%"Image set CBS001RGM-CBS010RGM from the Colocalization Benchmark Source
%(www.colocalization-benchmark.com) was used to validate colocalization."
\chapter{Related work}
There are many groups all over the world that have developed their own software to work on localization microscopy. This chapter shows the related work to give a better understanding what other concepts are used.\newline
There are several algorithms that estimate the PSF from the data:\newline
The DAOSTORM algorithm (\cite{DAO}) adapts algorithms from a software that is used to investigate crowded stellar fields. For spot localization a fit of multiple PSFs is used. This is done to small clusters of molecules and not globally. A model PSF is automatically generated from single PSFs in the data. DAOSTORM runs on the CPU.\newline
FPGA Estimator (\cite{simulated}) is an algorithm developed from a group at the Kirchoff-Institut f\"ur Physik in Heidelberg. It provides background subtraction by smoothing the pixels intensity over time, assuming Poisson distributed intensities. For high density data a method is used that sets all further pixel to zero once a local minimum is detected. A Gaussian estimator is used to determine the parameters of the Gaussian. The estimated scale can be compared with the given scale.\newline
QuickPALM (\cite{quickpalm}) uses the methods from rapidSTORM (\cite{rapidstorm}) which was published earlier. RapidSTORM uses a Spalttiefpass filter and the H\"ogborn "CLEAN" method for background suppression. For selecting the local maxima, a Gaussian function with scales either given or estimated from the data is fitted. An amplitude threshold is used to distinguish between signal and background pixels.\newline
There is also an algorithm that uses the given gain factor to transform the data to follow Poisson distributions. It is called GPUgaussMLE (\cite{alg3}).
A maximum likelihood estimate of the PSFs parameters is done. For background subtraction either a constant threshold or a dark imaged acquired from all frames is used. Two filters of different size are applied to the data to determine signal candidates. A patch around the candidates is then used for maximum likelihood estimation of the PSFs parameters. Based on the estimated parameters filters are applied that compare these values with given input values for the PSF width. Also the uncertainty of the estimation was used to filter the candidates. This algorithm runs on the GPU and can also be applied to 3D data.\newline
Most of the algorithms in contrast to SimpleSTORM use a maximum likelihood estimator to localize the maxima. Another example that uses the same method as simpleSTORM is called M2LE (\cite{M2LE}) which uses an user defined threshold for candidate selection and the median of a ROI for background suppression, an user specified ellipticity threshold which can be made intensity depended to discard candidates with an too high ellipticity. The PSFs position is determined using a maximum likelihood estimator, separated Gaussians are used for speed up. \newline
There are algorithms that provide their functionality as plugins for ImageJ or python packages. This makes it easier to implement these methods in ones workflow.\newline
There is the ImageJ plugin PeakFit (\cite{peakFit}) that uses a 2-dimensional Gaussian non-linear least squares Levenberg-Marquardt algorithm for fitting. The candidates are acquired by subtracting two filters of different sizes. The PSFs width can either be specified or estimated from the data. The fitted spots are filtered based on their signal-to-noise ratio. For high density data multiple Gaussians are fitted to the data. As an ImageJ plugin it can deal with any kind of data processable via ImageJ.\newline
PYME (\cite{PYME}) is a python package with functionalities that can be used for localization microscopy and other forms of widefield microscopy. Besides the STORM data analysis it also provides reconstruction and postprocessing algorithms. It is applicable to 2D and 3D data. For noise treatment a Gaussian-Poisson noise mixture model is used.\newline\newline

It becomes clear that there are only a couple of methods for the different task of STORM algorithms like background suppression or localization in use. Each software has a unique combination of these methods and different emphasis. No other software could be found that has a focus on a high user comfort. This is gap is filled by SimpleSTORM, which focuses on a low entry threshold for new users.


\input{ISBIChallange}
\chapter{Summary and outlook}
\section{Summary}
A new, easy to use and robust data processing algorithm was presented. It enables even users that are new to STORM imaging techniques to get reasonable results in a straight forward way. Understanding about several parameters and their influence on the algorithm is not necessary, as well as finding a good set of parameters.\newline
SimpleSTORM is also a powerful tool that allows tweaking of the parameters. Its performance has been shown in the ISBI localization microscopy challenge where SimpleSTORM showed good performance over all data sets.\newline
With the improved Colorcomposer software an easy to use and powerful tool for image reconstruction of multi-channel microscopy data is provided. The reconstructed images of different channels can be aligned automatically. A colocalization analysis can be performed.
\section{Outlook}
\subsection*{3d Storm}
STORM imaging techniques have been extended to capture 3-D structures up to a thickness of micrometers with super resolution. This will become more and more important in the future. There are different ways to get the information about the depth. One way is to use a cylindrical lens which results in point spread functions that are symmetrical if the fluorophore lies in the confocal plane, but more and more asymmetric the further away the captured spot lies off the confocal plane.\newline
Another way for depth estimation is to use normal lenses but use the information that the width of the PSF exceeds with greater distance of the spot to the confocal plane.\newline
Both methods might be implemented as only the part of the maximum detection has to be altered.
\subsection*{Improved method to detect maxima}
Instead of looking for the maxima in an upsampled image, a model for the point spread functions could be used to find the maxima. This would make it possible to find maxima that are too close together so that their maxima merge into only one maxima between them.\newline
This changes can be implemented without affecting the structure of the other parts of the SimpleSTORM software, just the maxima detection part must be changed.
\subsection*{Colorcomposer implemented in Storm-Gui}
The Colorcomposer could be implemented in the SimpleSTORM-GUI. The benefits beside increased performance are an easier and more direct way of the processing and the need for only one tool.
\subsection*{Combination of Wiener filter and Gaussian filter}
Depending on the spot density and the data attributes like the signal-to-noise ratio it might be favorable to combine the usage of a Wiener filter and a Gaussian filter to chose the filter that performs best on the data given.
\subsection*{Interactive parameter selection}
An unprocessed image is shown to the user. The user marks areas with spots of interest, specially the darkest spots that should be detected. Then the program sets the parameters in a way that at the user labeled spots will be detected in the end. This would be a good way to set the best alpha value or the signal-to-noise ratio limit.\newline

Alternatively one could set the parameters and get a direct feedback which points would be selected in the shown frame. If the results are not satisfying the parameters can be changed instantly.

\chapter{Appendix}


\listoffigures
\listoftables

\section{Additional tables of ISBI challenge results}

%\begin{table}[H]
\begin{center}
%\caption{Results for the main submission (with postprocessing)}
\begin{minipage}{\textwidth}
\captionof{table}{Results for the main submission for the high density data set (with postprocessing)}\label{reshd1}%
\begin{tabular}{lrrrr}
Dataset&Jaccard (in \%)&Precsision (in \%)& Recall (in \%) & RMSE (in nm)\\
HD1&40.44&100&41&40.13\\
HD2&30.84&100&31&63.18\\
HD3&12.55&100&13&61.8
\end{tabular} 
\end{minipage}
\end{center}
%\end{table}


%\begin{table}[H]
\begin{center}
\begin{minipage}{\textwidth}
\captionof{table}{Results for the high precission submission for the high density data set}\label{reshd2}
\begin{tabular}{lrrrr}
Dataset&Jaccard (in \%)&Precsision (in \%)& Recall (in \%) & RMSE (in nm)\\
HD1&37.62&100&38&40.23\\
HD2&28.37&100&28&60.70\\
HD3&12.66&100&13&62.58
\end{tabular}
\end{minipage}
\end{center}
%\end{table}

\begin{center}
\begin{minipage}{\textwidth}
\captionof{table}{Results for the high score submission for the high density data set (without postprocessing)}\label{reshd3}
\begin{tabular}{lrrrr}
Dataset&Jaccard (in \%)&Precsision (in \%)& Recall (in \%) & RMSE (in nm)\\
HD1&37.96&100&38&40.22\\
HD2&29.73&94&30&63.33\\
HD3&13.00&99&13&63.95
\end{tabular}
\end{minipage}
\end{center}


\begin{center}
\begin{minipage}{\textwidth}
\includegraphics[width = 0.88\textwidth]{pictures/diagrammsChallenge/MeanHighDensityJaccardCropped.pdf}
	\captionof{figure}{Results for high density data sets. The Jaccard indices are averaged over all three data sets. For this score higher is better.}
	\label{meanJaccardHighDensity}
	\end{minipage}
\end{center}

\begin{center}
\begin{minipage}{\textwidth}
\includegraphics[width = 0.88\textwidth]{pictures/diagrammsChallenge/MeanHighDensityRSMECropped.pdf}
	\captionof{figure}{Results for high density data sets. Averaged RSME scores over all three datasets. For this score lower is better.}
	\label{meanRSMEHighDensity}
	\end{minipage}
\end{center}



\begin{center}
\begin{minipage}{\textwidth}
\captionof{table}{Results for the main submission for low density data sets(with postprocessing)}
\begin{tabular}{lrrrr}
Dataset&Jaccard (in \%)&Precsision (in \%)& Recall (in \%) & RMSE (in nm)\\
LS1&86.12&100&86&26.32\\
LS2&69.64&97&71&37.43\\
LS3&47.48&99&48&54.2
\end{tabular}\label{resls1}
\end{minipage}
\end{center}


\begin{center}
\begin{minipage}{\textwidth}
\captionof{table}{Results for the high precission submission for low density data sets}\label{resls2}
\begin{tabular}{lrrrr}
Dataset&Jaccard (in \%)&Precsision (in \%)& Recall (in \%) & RMSE (in nm)\\
LS1&83.39&100&83&27.91\\
LS2&63.21&100&63&39.57\\
LS3&40.82&100&41&49.55
\end{tabular}
\end{minipage}
\end{center}

\begin{center}
\begin{minipage}{\textwidth}
\captionof{table}{Results for the high score submission for low density data sets (without postprocessing)}\label{resls3}
\begin{tabular}{lrrrr}
Dataset&Jaccard (in \%)&Precsision (in \%)& Recall (in \%) & RMSE (in nm)\\
LS1&87.23&99&88&28.10\\
LS2&65.95&89&72&44.60\\
LS3&40.82&96&48&54.40
\end{tabular}
\end{minipage}
\end{center}



\begin{center}
\begin{minipage}{\textwidth}
\includegraphics[width = 0.88\textwidth]{pictures/diagrammsChallenge/MeanLowDensityJaccardCropped.pdf}
	\captionof{figure}{Results for low density data sets. The Jaccard indices are averaged over all three data sets. For this score higher is better.}
	\label{meanJaccardLowDensity}
\end{minipage}
\end{center}

\begin{center}
\begin{minipage}{\textwidth}
\includegraphics[width = 0.88\textwidth]{pictures/diagrammsChallenge/MeanLowDensityRSMECropped.pdf}
	\captionof{figure}{Results for low density data sets. Averaged RSME scores over all three datasets. For this score lower is better.}
	\label{meanRSMELowDensity}
\end{minipage}
\end{center}

\begin{center}
\begin{minipage}{\textwidth}
\includegraphics[width = 0.88\textwidth]{pictures/diagrammsChallenge/MeanLowDensityPrecisionCropped.pdf}
	\captionof{figure}{Results for low density data sets. Averaged Precision over all three datasets. For this score higher is better. The y axis is broken to show the differences better.}
	\label{meanPrecisionLowDensity}
\end{minipage}
\end{center}

\begin{center}
\begin{minipage}{\textwidth}
\includegraphics[width = 0.88\textwidth]{pictures/diagrammsChallenge/MeanRankLowDensityCropped.pdf}
	\captionof{figure}{Averaged rank over Jaccard index and RSME score for all low density data sets. Lower ranks are better.}
	\label{meanRankLow}
\end{minipage}
\end{center}

\begin{center}
\begin{minipage}{\textwidth}
\includegraphics[width = 0.88\textwidth]{pictures/diagrammsChallenge/MeanRankHighDensityCropped.pdf}
	\captionof{figure}{Averaged rank over Jaccard index and RSME score for all high density data sets. Lower ranks are better}
	\label{meanRankHigh}
	\end{minipage}
\end{center}


\bibliographystyle{te}

\bibliography{research}
\newpage
\thispagestyle{empty}
\setlength{\parindent}{0em}

Erkl\"{a}rung:\par
\vspace{3\baselineskip}
Ich versichere, dass ich diese Arbeit selbstst\"{a}ndig verfasst habe und keine
anderen als die angegebenen Quellen und Hilfsmittel benutzt habe.\par
\vspace{5\baselineskip}
Heidelberg, den 8. Juni 2013\hspace{3cm}\dotfill

\end{document}
