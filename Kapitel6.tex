\chapter{CCD camera}

\section{Image acquisition}
\subsection{Photon sources with shot noise}
The emission of photons is a random process that occures at unpredictible times. Therefore the number of photons passing through a plane is never constant but varies around some average value. The phenomena, that one can never determine exactly how many photons should hit the sensor chip of a CCD camera for example, is called shot noise. It playes a major role if the total number of photons is low, as from dark sources or with short exposure times of the camera.
\subsection{Quantum efficiency}
Quantum efficiency describes the fraction of photons that create a detectable electron in a sensor chip. The quantum efficiency is dependent of the wavelenght of the incoming photon. Photons with energies below the band gap can't produce a free electron that can be detected. The quantum efficiency has a maximum basically caused by two effects. The higher the photons energy the higher the kinetic energy of the freed electron, but it is absorbed earlier and can therefore recombinate with a electron hole more likely.
\subsection{Gain}
There are two different gain factors involved in the capturing process of a camera. First the electric signal for each pixel might be amplified. And there is also a gain factor that describes the proportionality between collected electrons and the digital number that is associated with.
\subsection{Readout noise}
The origin of readout noise is the amplifier. The aplification is never perfect, this means the exact number of electrons at the end of the amplification has some variation around the expected linearly increased value. There might also be some random signals of the electronics that add to the "true" signal. The readout noise is independent of the exposure time.
\subsection{Dark current noise}
Dark current noise is generated by the thermal movement of the atoms in the sensor chip. The movement of molecules and atoms is dependent of the temperature of the material, because of that dark current noise depends strongly on the temperature of the chip and can be reduced by cooling. Dark current noise generates electrons in the bins of each pixel even with closed shutter it is constantly increasing with time and follows Poisson statistics.
\subsection{Quantisation}
The signal must fit into the output color depth. It has to be rounded or truncated to fit in. This process introduces errors that can be seen as additional noise that is dependent on the intensity of the signal. High intensities are disturbed less relative to low intensies.
