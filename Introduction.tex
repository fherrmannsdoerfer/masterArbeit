\chapter{Introduction}
How does a virus reproduce? Which proteins are involved in chemical synapses? How do proteins interact? These questions and many more arise in biology and related sciences. Microscopy is a powerful tool to answer these questions, because it gives information about the distribution of proteins within a cell. However light microscopy is limited in spatial resolution due to limited diffraction, as described by \cite{Abbe} \footnote{There is a fundamental limit for the resolution of optical systems. It is roughly half the wavelength of the light which is used for microscopy. With green light of 500 nm this means the minimal resolution achievable is 250 nm}. Recently there have been developed different methodes to increase the resolution of light microscopy beyond the diffraction limit. To name a few: Photoactivated Localization Microscopy (PALM) \cite{Palm}, Stochastic Optical Reconstruction Microscopy (STORM) \cite{Storm} or Stimulated Emission Depletion STED \cite{sted}. For these techniques many images with sparsely distributed signals, coming from fluorophors which are blured by diffraction are used. Because the signals are sparse the center of each signal can be determined with sub pixel accuracy. This is not possible for one image with all signals, which would be blurred and undistinguishable. There are different ways to stain the biological sample with fluorophors. One way is to use antibodies which bind to specific targets within a cell. The antibody is either directly linked to a fluorophore or two antibodies are used, where the first binds to the biological structre and the second with the fluorophore attached to the first antibody. 
\newline
For this thesis only the STORM medode is considered. It can also be used to investigate the distribution of different proteins within a cell. To do so each protein is labeled with different fluorophores. Images can be aquired showing just signal from one kind of fluorophore. However, to be able to seperate the different signals from the flourophores, the emission spectrum must be distinct. This leads to cromatic aberration which results in images that are distorted and thus can not be aligned easily.\newline 
Many research groups have developed their own software to process STORM data sets. But most of these programs usually need many parameters, like the point spread functions width or the camera parameters, must be set by the user to get reasonable results, therefore these programs are difficult to use for somebody who is not familiar with image processing or does not know the parameters influence on the results.\newline

One key question of this thesis is how to build software that is easy to use even without prior information about the data or knowledge of image processing? \newline

Our answer to this question is SimpleStorm a software that calibrates itself. It estimates the camera parameters and the width of the point spread function of the fluorophores. In contrast to many other software applications, no threshold is needed.\newline
If more than one kind of fluorophore was used to stain the sample, the resulting images are distorted relative to each other. The colorcomposer tool was developed to do this task automatically and to analyse the aligned images. The colocalisation of the molecules of the different channels is a useful measure in biology to determine whether or not near molecules might interact. The colorcomposer software calculates both global and local measures of localisation.


